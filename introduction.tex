% !TEX root = ./multilinear.tex
\section{Introduction}

Many problems in graph mining and social network analysis can be reduced to questions
about different kinds of subgraphs; two important classes of such problems, which 
are the focus of our paper, are:
(1) \emph{Detecting subgraphs}, such as paths and trees, of a given size $k$---these are used
for characterizing different kinds of networks, especially in biological models
\cite{alon2008biomolecular, huffner2008algorithm}.
(2) \emph{Anomaly detection in networked data using graph scan statistics}, which involves finding
connected subgraphs of size $k$, optimizing different kinds of objectives \cite{Speakman-14,leiserson2015pan,
hansen2016finding, \iftoggle{refsfull}{neil2013scan,} chen2014non}.
This problem arises in a number of applications, such as social network analysis, 
epidemiology, finance, and bio-surveillance. We refer to the survey by 
Akoglu et al. \cite{akoglu2014graph} for a detailed discussion on different approaches 
to graph anomaly detection.

The problems of subgraph detection and scan statistics are NP-hard, in general, making them 
computationally very challenging. Various sequential heuristics have been proposed for
both these problem classes, e.g.,
\cite{alon2008biomolecular, huffner2008algorithm, Speakman-14,leiserson2015pan,
hansen2016finding, \iftoggle{refsfull}{neil2013scan,} chen2014non, cheng:kdd12}. However, these
do not scale to large instances, motivating the need for parallel implementations.
There has been a fair amount of work on parallel algorithms for various kinds of subgraph
problems, e.g., \cite{zhao2012sahad, slota:icpp13, slota:ipdps14, arif:cikm13, 
schmidt2009scalable, zhao2016parallel, aparicio:ispa14, du:mcd09}, but very little for
anomaly detection using graph scan statistics \cite{zhao2016parallel}.


Most of these parallel algorithms involve heuristics based on partitioning, pruning, and 
enumeration for speeding up the computations to fairly large graphs---e.g., for 
maximal cliques \cite{schmidt2009scalable, zhao2016parallel, aparicio:ispa14, cheng:kdd12, du:mcd09}.
One of the few rigorous approaches for subgraph counting that has been used for
developing parallel algorithms is based on the ``color coding'' technique
\cite{alon2008biomolecular, alon1995color}, which gives a rigorous \emph{fixed parameter tractable} algorithm\footnote{This means the running time is $O(f(k)n^c)$, where
$n$ is the number of nodes, $c$ is a constant, and $f(\cdot)$ is a function of some other
parameter---in this case, it is the size of the subgraph \cite{downey}.}
for finding and counting subgraphs. Color coding is a dynamic programming algorithm,
and multiple approaches have been developed for parallelizing it 
\cite{zhao2012sahad, slota:icpp13, slota:ipdps14}; the FASCIA algorithm of \cite{slota:icpp13, slota:ipdps14}
which is implemented in MPI is the state-of-the-art, and can handle trees
of size up to 12 in graphs with millions of nodes. 
Color coding has also been used to develop a
sequential algorithm for graph scan statistics \cite{cadena:sdm17}, but no parallel
adaptations exist so far.
A big challenge of the color coding technique is the memory overhead, since
it scales as $2^k$, where $k$ denotes the subgraph size, which limits further
scaling to larger graphs or subgraphs.  

In this paper, we develop a novel approach for designing highly scalable parallel
algorithms for both the problem classes mentioned above, by adapting
powerful algebraic techniques for
detecting multilinear terms in a multivariate polynomial developed by 
Koutis \cite{koutis:icalp08} and Williams \cite{williams2009finding}.
Our contributions are:

\noindent
\textbf{1. Efficient algorithms for parallel multilinear detection
and applications to subgraph analysis.}
We develop a distributed MPI based algorithm for detecting 
multilinear terms with $k$ variables of the form $x_{i_1}x_{i_2}\ldots x_{i_k}$ (i.e., a term in
which all variables have exponent 1) in a multivariate polynomial
$P(x_1,\ldots,x_n)$---this is referred to as the $k$-Multilinear Detection (\textsc{$k$-MLD}) problem
\cite{koutis:icalp08, williams2009finding}. The problem of finding paths and trees can
be reduced to the \textsc{$k$-MLD} problem \cite{koutis:icalp08}. We also show that the
network scan statistics problem can also be reduced to the \textsc{$k$-MLD} problem.
Consequently, our parallel multilinear detection algorithm leads to parallel algorithms
for both these problems. Our methods provably
have better space and running time complexity compared to the color coding technique.
The space complexity is $O(kn)$ compared to $O(2^kn)$, whereas the running time complexity is proportional to $O(2^k)$, compared to $O(2^ke^k)$
\cite{zhao2012sahad, slota:icpp13, slota:ipdps14}.

\noindent
\textbf{2. Experimental results}. We evaluate our results on a number of real and synthetic networks 
with up to 250 million edges and subgraph sizes up to $k=18$. The reduced memory footprint
allows us to scale to paths of size 18, which has not been done before.  Our algorithms for
both problems show reasonable scaling up to 512 processors. The running time grows linearly
with the network size and as $2^k$ for the subgraph size $k$.

\noindent
\textbf{3. Comparison with prior methods.}
Our algorithm for finding paths gives over two orders of magnitude improvement in time
compared to FASCIA, the state of the art method based on color coding \cite{slota:icpp13, slota:ipdps14}.
There is only one parallel algorithm for network scan statistics, which is based on pruning heuristics,
and our method gives several order of magnitude improvement over it.

\noindent
\textbf{4. Comparison with Giraph.}
We compare our methods with recent parallel graph programming models, Giraph and Spark.
Our results show that neither of these models scale well for these problems; further,
Spark has worse performance than Giraph.


%we study a novel approach for a broad class of subgraph analysis problems
%in massive networks based on powerful algebraic techniques. This involves formalizing these
%problems in terms of detecting multilinear terms in a multivariate polynomial, based on 
%very powerful techniques developed by Koutis \cite{koutis:icalp08} and Williams \cite{williams2009finding}.
%However, only sequential algorithms have been developed so far \cite{bjorklund:esa14}, which do not
%scale beyond networks with $\sim 10^3-10^4$ nodes.
%We develop a novel approach for parallel multilinear detection, which allows us to
%solve a broad class of subgraph analysis and anomaly detection problems with the same tools.
%
%Both multilinear detection and subgraph counting using color coding are examples of
%\emph{fixed parameter tractable} algorithms, which run in time $O(f(k)n^c)$, where
%$n$ is the number of nodes, $c$ is a constant, and $f(\cdot)$ is a function of some other
%parameter---in this case, it is the size of the subgraph. Most of these problems are
%NP-complete, in general, and such algorithms are one way to handle the hardness; this is
%referred to as \emph{parameterized complexity} (see \cite{downey} for an excellent discussion
%of such methods for many fundamental NP-hard problems).
%
%In this paper, we study parallel algorithms for multilinear detection and their applications
%to different subgraph analysis problems, using multiple models of parallel graph computation.
%Our contributions are:

