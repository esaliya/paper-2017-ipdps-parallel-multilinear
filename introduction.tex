% !TEX root = ./multilinear.tex
\section{Introduction}

Many problems in graph mining and social network analysis can be reduced to questions
about different kinds of subgraphs; two important classes of such problems, which
are the focus of our paper, are:
(1) \emph{Detecting and counting subgraphs}, such as paths and trees, of a given size $k$---these are used
for characterizing different kinds of networks, especially in biological models
\cite{alon2008biomolecular, huffner2008algorithm}.
(2) \emph{Anomaly detection in networked data using graph scan statistics}, which involves finding
connected subgraphs of size $k$, optimizing different kinds of objectives \cite{Speakman-14,leiserson2015pan,
hansen2016finding, \iftoggle{refsfull}{neil2013scan,} chen2014non}---this
arises in a number of applications, such as social network analysis,
epidemiology, finance, and bio-surveillance (see \cite{akoglu2014graph} for a survey).

These problems are computationally very challenging. For instance, exact detection of
paths is NP-hard and the corresponding counting problem is \#P-hard. Development
of parallel algorithms for these problems has been an active area of research.
Many parallel algorithms exist for counting local subgraphs, such as triangles,
e.g., \cite{arif:cikm13, schmidt2009scalable, aparicio:ispa14, du:mcd09}.
Finding trees is much harder, and a number of heuristics have been developed. One
of the few techniques that gives rigorous approximation guarantees is
\emph{color coding} \cite{alon2008biomolecular, alon1995color, huffner2008algorithm}.
Parallel adaptations have been developed using MapReduce \cite{zhao2012sahad} and
MPI \cite{slota:icpp13, slota:ipdps14}. The MPI based
FASCIA algorithm \cite{slota:icpp13, slota:ipdps14} is the state-of-the-art in terms of
counting trees in massive networks---it is able to scale to trees with up to 12 vertices,
and billion edges. However, it seems very hard to scale the color coding
method to larger subgraphs, even on smaller networks. The main reason is that the
time and space complexity of the color coding technique
both scale as $2^k$, where $k$ denotes the subgraph size.
In this paper, we take the first steps towards beating this bound,
which has remained a significant open problem since \cite{slota:ipdps14}.
We approach involves parallelization of a powerful algebraic technique for
detecting multilinear terms in a multivariate polynomial, developed by
Koutis \cite{koutis:icalp08} and Williams \cite{williams2009finding}.



Optimizing network scan statistics leads to challenging optimization problems.
Color coding has also been used to develop the first method with rigorous approximation
guarantees \cite{cadena:sdm17}; however, this has not been parallelized because of
its high memory overhead. In \cite{cadena:bigdata17}, we have developed a
parallel adaptation of the multilinear detection technique using Giraph. However,
this does not scale beyond networks with 40 million edges.

In this paper, we develop a distributed algorithm for multilinear detection,
which immediately leads to highly scalable algorithms for both paths and trees,
and network scan statistics.
Our contributions are:


\noindent
\textbf{1. \ouralgo{}: Distributed algorithm for multilinear detection
and applications to subgraph analysis.}
We develop \ouralgo{}, a distributed MPI based algorithm for finding
paths and trees through detection of
multilinear terms with $k$ variables of the form $x_{i_1}x_{i_2}\ldots x_{i_k}$ (i.e., a term in
which all variables have exponent 1) in a multivariate polynomial
$P(x_1,\ldots,x_n)$ (which is referred to as the $k$-Multilinear Detection (\textsc{$k$-MLD}) problem
\cite{koutis:icalp08, williams2009finding}). The sequential algorithm uses a matrix representation
of a group algebra (as will be described later), and its structure lends itself to a natural
parallelization.  We give rigorous bounds on the performance in terms of the time and
space complexity, which scale as
$O(2^k)$ and $O(k)$, respectively, compared with $O(2^ke^k)$ and $O(2^k)$ for
color coding, respectively \cite{zhao2012sahad, slota:icpp13, slota:ipdps14}.
For random graphs, we show a rigorous scaling with $N$, the number of processors for $N\leq 2^k$.

\noindent
\textbf{2. Cache optimization and weak scalability}. %%Parallelization of the \textsc{$k$-MLD} problem {\color{red} TODO}
Our algorithm partitions the graph $G$ into $N_1$ parts. The computation involves $2^k$
iterations, and $N_1$ processors together perform one iteration---this allows us to schedule
$N/N_1$ such computations to occur in parallel. The total compute time exhibits good weak scaling.
Additionally, our data structures for supporting Galois field operations
during the iterations support a temporal cache locality, which actually leads to a reduction
in the compute time as $N_1$ increases. On the other hand, the communication cost increases
with $N_1$, leading to an optimal value of $N_1$ for the best performance.




\noindent
\textbf{3. Experimental results}. We evaluate our results on a number of real and synthetic networks 
with up to 250 million edges and subgraph sizes up to $k=18$. The reduced memory footprint
allows us to scale to paths of size 18, which has not been done before.  Our algorithms for
both problems show reasonable scaling up to 512 processors. The running time grows linearly
with the network size and as $2^k$ for the subgraph size $k$.

\noindent
\textbf{4. Comparison with prior methods.}
Our algorithm for finding paths gives over two orders of magnitude improvement in time
compared to FASCIA, the state of the art method based on color coding \cite{slota:icpp13, slota:ipdps14}.
Our algorithm for scan statistics improves on the Giraph based implementation
\cite{cadena:bigdata17} by over an order of magnitude, and scales to significantly larger networks.


%we study a novel approach for a broad class of subgraph analysis problems
%in massive networks based on powerful algebraic techniques. This involves formalizing these
%problems in terms of detecting multilinear terms in a multivariate polynomial, based on 
%very powerful techniques developed by Koutis \cite{koutis:icalp08} and Williams \cite{williams2009finding}.
%However, only sequential algorithms have been developed so far \cite{bjorklund:esa14}, which do not
%scale beyond networks with $\sim 10^3-10^4$ nodes.
%We develop a novel approach for parallel multilinear detection, which allows us to
%solve a broad class of subgraph analysis and anomaly detection problems with the same tools.
%
%Both multilinear detection and subgraph counting using color coding are examples of
%\emph{fixed parameter tractable} algorithms, which run in time $O(f(k)n^c)$, where
%$n$ is the number of nodes, $c$ is a constant, and $f(\cdot)$ is a function of some other
%parameter---in this case, it is the size of the subgraph. Most of these problems are
%NP-complete, in general, and such algorithms are one way to handle the hardness; this is
%referred to as \emph{parameterized complexity} (see \cite{downey} for an excellent discussion
%of such methods for many fundamental NP-hard problems).
%
%In this paper, we study parallel algorithms for multilinear detection and their applications
%to different subgraph analysis problems, using multiple models of parallel graph computation.
%Our contributions are:

