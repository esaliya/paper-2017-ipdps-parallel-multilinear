% !TEX root = ./multilinear.tex
\section{Preliminaries}
\label{sec:prelim}

\subsection{Problem Formulation and Notation}
We will focus on the following two classes of problems in this paper, but our
approach can be extended more broadly.

\noindent
%\textbf{Finding Paths and Trees}
\subsubsection{Finding Paths and Trees}
Given a graph $G=(V, E)$ with $n=|V|$, $m=|E|$, and a subgraph $H=(V^H, E^H)$, with $k=|V^H|$, find a mapping $f:V^H\rightarrow V$,
such that $(i, j)\in E_H$ \emph{only if} $(f(i), f(j))\in E$. Such a mapping is referred to as a \emph{non-induced embedding} of $H$ in $G$.

\begin{problem} ($k$-Tree)
\label{prob:trees}
Given a weighted graph $G=(V, E)$ with a weight vector $\mathbf{w}$, and a tree
denoted by $H=(V^H, E^H)$ with $|V^H|=k$, the objective is to determine if there exists
an embedding of $H$ in $G$.
\end{problem}

We will consider the following approximate version of Problem \ref{prob:trees}: determine if a non-induced embedding exists with probability at least $1-\epsilon$, where $\epsilon\in(0, 1)$ is a parameter.
Other common variants of this problem are: (1) counting all embeddings, and (2) finding a maximum
weight embedding in a weighted version of the graph; our approach can be extended to
these variants.


\noindent
%\textbf{Anomaly Detection Using Graph Scan Statistics}
\subsubsection{Anomaly Detection Using Graph Scan Statistics}
We use the notation of \cite{cadena:sdm17}. We assume each node $v\in V$ has two values,
which vary with time (we will not show the time to avoid complicating the notation):
(1) a \emph{baseline count}, $b(v)$, which indicates the count that we
expect to see at the node $v$---e.g., the number of people in a county corresponding to node $v$---and
(2) an \emph{event count} or \emph{weight}, $w(v)$, which indicates how many occurrences of an event
of interest are seen at the node---e.g., the number of cases of a disease in a county.

Graph scan statistics are among the most commonly used methods for detecting anomalies or ``hotspots" in
network data \cite{Speakman-14,leiserson2015pan, hansen2016finding, neil2013scan, chen2014non}.
Informally, this approach formalizes anomaly detection as a hypothesis testing problem.
Under the null hypothesis $H_0$, it is \emph{business as usual}, and the event counts for all nodes are generated proportionally to their baseline counts. Under the alternative hypothesis $H_1(S)$, counts of a majority of
the vertices are generated (again) with rate proportional to the baseline counts, but there exists a small connected subset
$S \subseteq V$ of vertices for which the counts are generated at a higher rate than expected.
Then, the goal is to find a set of vertices $S$ that maximizes an appropriate scan statistic function $F(S)$, typically a log-likelihood ratio that compares event counts to baseline counts. We define a scan statistic in terms of the event and baseline counts of a node set:
$$
F(S) = F(W(S), B(S), \mathbf{\theta}),
$$
where $W(S) = \sum_{v \in S} w(v)$ is the total event count or \emph{weight} of $S$, $B(S) = \sum_{v \in S} b(v)$ is the baseline count of the set, and $\theta$ represents possible additional arguments to $F$.

The graph anomaly detection problem can be posed as the following constrained optimization problem.

\begin{problem}
\label{prob:macs}
Given a graph $G=(V, E)$, a scan statistic $F(\cdot)$, the associated counts for vertices---$\mathbf{w}$ and $\mathbf{b}$---and a parameter $k$, find a connected subset $S\subseteq V$ that maximizes $F(S) = F(W(S), B(S), \theta)$ with $B(S) \leq k$.
\end{problem}

Problem \ref{prob:macs} is NP-hard \cite{cadena:sdm17}. We consider the following approximate version: find a subset $S$ such that $B(S)\leq k$ and $F(S)$ equals the optimum, with probability at least $1-\epsilon$, where $\epsilon\in(0, 1)$ is a parameter. Other variants of the problem include finding the top--10 subgraphs with highest $F(S)$, and finding all subgraphs with $F(S)$ above some threshold.

% \noindent
% \textbf{Our approach.} Problems \ref{prob:trees} and \ref{prob:macs} are seemingly very different,
% and they are computationally very challenging. Problem \ref{prob:macs}
% has been shown to be NP-hard in general \cite{cadena:sdm17}. Finding paths of length at least $k$
% is NP-hard when $k$ is large. We will focus on the setting where $k$ is small, e.g., $O(\log{n})$.
% The only prior approach that has given parallel algorithms with rigorous bounds for these two problems involves the use of the technique called \emph{color coding} (\cite{alon2008biomolecular, huffner2008algorithm,
% alon1995color} for Problem \ref{prob:trees} and \cite{cadena:sdm17} for Problem \ref{prob:macs}).
% However, color coding has a significant memory overhead, and we propose a new approach
% based on an algebraic technique that involves detecting multilinear terms in multivariate polynomials.
% We first discuss the basic theoretical concepts needed for this approach, and then the seminal
% algorithm of \cite{DBLP:journals/talg/KoutisW16, williams2009finding} for multilinear detection.
% Our approach involves two steps:
% \begin{itemize}
% \item
% We develop an MPI-based parallel algorithm for multilinear detection of any multivariate
% polynomial.%, represented as a directed acyclic graph (DAG) (Section \ref{sec:proposed}).
% \item
% We show that both our problems can be reduced to multilinear detection for suitably defined
% polynomials, and thus can be solved in parallel. %The parallel algorithm in Section \ref{sec:proposed} is described for any multivariate polynomial, represented as a DAG, and serves as an abstraction.% for describing both algorithms. 
% We discuss how the specific
% polynomials are constructed for our problems.
% \end{itemize}


\section{$k$-Multilinear Detection and Sequential Algorithm}
\label{sec:kMD}
We describe the sequential multilinear detection algorithm. We start with some introduction to group algebras and Galois fields and end with the sequential algorithm for finding $k$-paths.
Because of the limited space, we only describe the main ideas here and refer to \cite{koutis:icalp08, williams2009finding} for more details. 

Let $X = x_1, \ldots,x_n$ be a set of variables, and let $P(X)$ be a polynomial, which is a sum 
of monomials on $X$. We will denote $P(X)=\sum_S \Pi_{i\in S} x_i$ as a monomial, where
the sum is over multisets $S$.  An example of a polynomial on six variables is 
$P(x_1,x_2,x_3,x_4, x_5, x_6) = x_1^2x_2 + x_2x_3x_4 + x_3x_4x_5 + x_5x_6$. 
A monomial is called \emph{multilinear} or \emph{square-free} if all its variables 
have exponent 1, and its \emph{degree} is the sum of the exponents of all its variables. 
For instance, in the example above, $x_2x_3x_4$, $x_3x_4x_5$, and $x_5x_6$ are multilinear monomials, but $x_1^2x_2$ is not multilinear. 
Given variables $X = x_1, \ldots ,x_n$ and a polynomial $P(X)$, the goal in 
the $k$-Multilinear Detection (\textsc{$k$-MLD}) problem
is to decide whether or not $P(X)$ has a multilinear monomial of degree exactly $k$. 

We note that the polynomial $P(X)$ may have an arbitrary number of terms---i.e., exponential on the size of $n$---therefore, the problem is not as simple as writing the polynomial explicitely and checking each term. Rather, we assume that $P(X)$ is given succintly in a recursive form, and the ``yes"/``no" decision has to be made without unrolling this recursion.
%Figure \ref{fig:dag} illustrates a circuit representation of the polynomial $P(\cdot)$ above.
In general, we also have a weight $w_S$ for each multinomial $\Pi_{i\in S} x_i$. Formally, we have the following problem.

%\begin{figure}[h]
%\includegraphics[width=0.5\textwidth]{img/dag3_fixed.pdf}
%\caption{
%\small
%The polynomial $P(x_1,x_2,x_3,x_4, x_5, x_6) = x_1^2x_2 + x_2x_3x_4 + x_3x_4x_5 + x_5x_6$
%represented as a circuit with multiplication ($\cdot$) and addition ($+$) gates, which is a DAG. 
%Each variable $x_i$ is assigned the element $v_0+v_i\in\mathbb{Z}_2[\mathbb{Z}_2^2]$,
%for randomly chosen $v_i\in \mathbb{Z}_2^2$. The computations at each gate
%are in the group algebra, and are shown in blue. The circuit evaluates to the element $J$.
%\vspace{-0.2in}
%}
%\label{fig:dag}
%\end{figure}

\begin{problem} (\textsc{$k$-MLD} problem)
Given a polynomial $P(\cdot)$ defined recursively,
%represented by a DAG $G=(V, E)$, 
in which each monomial
has degree at most $k$ and weight weight $w_S$, determine:
(1) if $P(\cdot)$ has a multilinear term of degree $k$, and 
(2) the maximum weight of any multilinear term, if one exists.
\end{problem}

\subsection{Group Algebras}
\label{sec:grpalgebra}
We discuss some notation from group algebras that is crucial for the paper. 
Let $\mathbb{Z}_2^k$ be the group formed by all the $k$-dimensional binary vectors, and define the group multiplication operation as entry-wise XOR. For example, $\mathbb{Z}_2^2$ consists of the vectors $v_0 = (0, 0), v_1 = (0, 1), v_2 = (1, 0), v_3 = (1, 1)$. We note that $v_0$ is the multiplicative identity of the group, and each element is its own multiplicative inverse: $v_i \cdot v_i = v_0$. Now, we define a group algebra $\mathbb{Z}_2[\mathbb{Z}_2^k]$. Each element in the group algebra is a sum of elements from $\mathbb{Z}_2^k$ with coefficients from $\mathbb{Z}_2$ (i.e., either 1 or 0):
$
\sum_{v\in \mathbb{Z}_2^k} a_v v,
$
where $a_v \in \{0,1\}$. The addition operator of the group algebra is
{\scriptsize
$$
\sum_{v\in \mathbb{Z}_2^k} a_v v + \sum_{v\in \mathbb{Z}_2^k} b_v v = \sum_{v\in \mathbb{Z}_2^k} (a_v + b_v) v,
$$}
where the addition of the coefficients is modulo 2, and the multiplication is defined as
{\scriptsize
$$
\left(\sum_{v\in \mathbb{Z}_2^k} a_v v\right)\left(\sum_{u\in \mathbb{Z}_2^k} b_u u\right) = \sum_{v\in \mathbb{Z}_2^k} (a_v \cdot b_u) (v\cdot u).
$$}


\begin{mdframed}
\scriptsize{
\noindent
\textbf{Example.} For $k=2$, the group algebra $\mathbb{Z}_2[\mathbb{Z}_2^k]$ has
$2^{2^2}=16$ elements, such as
\[
x_1=
0\cdot
\begin{bmatrix}
0\\
0
\end{bmatrix}
+ 1\cdot
\begin{bmatrix}
0\\
1
\end{bmatrix}
+
1\cdot
\begin{bmatrix}
1\\
0
\end{bmatrix}
+ 0\cdot
\begin{bmatrix}
1\\
1
\end{bmatrix}
, \mbox{ which we also write as }
\begin{bmatrix}
0\\
1
\end{bmatrix}
+
\begin{bmatrix}
1\\
0
\end{bmatrix}
\]


\[
x_2 =
\begin{bmatrix}
0\\
0
\end{bmatrix}
+
\begin{bmatrix}
1\\
0
\end{bmatrix}
\]

We have 
\[
x_1 + x_2 = 
\begin{bmatrix}
0\\
0
\end{bmatrix}
+
\begin{bmatrix}
0\\
1
\end{bmatrix}
+ 2
\begin{bmatrix}
1\\
0
\end{bmatrix}
=
\begin{bmatrix}
0\\
0
\end{bmatrix}
+
\begin{bmatrix}
0\\
1
\end{bmatrix}
\]

\[
x_1x_2 = 
\left(
\begin{bmatrix}
0\\
1
\end{bmatrix}
+ 
\begin{bmatrix}
1\\
0
\end{bmatrix}
\right)\cdot 
\left(
\begin{bmatrix}
0\\
0
\end{bmatrix}
+
\begin{bmatrix}
1\\
0
\end{bmatrix}
\right) =
\begin{bmatrix}
0\\
0
\end{bmatrix}
+
\begin{bmatrix}
0\\
1
\end{bmatrix}
+
\begin{bmatrix}
1\\
0
\end{bmatrix}
+
\begin{bmatrix}
1\\
1
\end{bmatrix}
\]
It is easy to check that
\[
x_1^2x_2 = 
0\cdot
\begin{bmatrix}
0\\
0
\end{bmatrix}
+ 0\cdot
\begin{bmatrix}
0\\
1
\end{bmatrix}
+ 0\cdot
\begin{bmatrix}
1\\
0
\end{bmatrix}
+ 0\cdot
\begin{bmatrix}
1\\
1
\end{bmatrix}
= \mathbf{\bar{0}} \text{ (additive identify)}
\]
}
\end{mdframed}

\subsection{Sequential algorithm for Multilinear Detection} 
\label{sec:seq}
We briefly discuss the algorithm of Koutis \cite{koutis:icalp08}, which forms the basis of our paper. An important property is that for any $v_i \in \mathbb{Z}_2^k$, the square of the term $(v_0 + v_i) \in \mathbb{Z}_2[\mathbb{Z}_2^k]$ evaluates to 0:
{\small
$$
(v_0 + v_i)^2 = v_0^2 + 2(v_0\cdot v_i) + v_i^2 = v_0 + (0\mod 2)v_i + v_0 = 2v_0 = 0.
$$}

%The main idea in the algorithm of \cite{koutis:icalp08} is that, if we evaluate a polynomial over the ``right" algebra, monomials that have square terms evaluate to 0, and the remaining terms, which are multilinear, do not cancel out, with some probability. Then, a polynomial $P(X)$ has a $k$ multilinear term if $P(X) \neq 0$. 

We can show that, if we choose a $v_i \in \mathbb{Z}_2^k$ uniformly at random and set $x_i = v_0 + v_i$, then a multilinear monomial does \textbf{not} evaluate to $\bar 0$ with some probability, whereas a monomial with squares is always $\bar 0$ (as in the box above).
The algorithm was later refined in \cite{williams2009finding} by evaluating the polynomial over the group algebra $GF(2^{3 + \log_2k})[\mathbb{Z}_2^k]$, where $GF(p)$ is the Galois field of order $p$ \cite{mullen2007finite}, and this is the version that we implement. But, to simplify the discussion below, we assume that we are working on the group algebra $\mathbb{Z}_2[\mathbb{Z}_2^k]$. 

A polynomial $P(x_1,\ldots,x_n)$ with variables from $\mathbb{Z}_2[\mathbb{Z}_2^k]$ can be evaluated in time $O(2^k poly(n))$ and space $O(2^kpoly(n))$, resulting in Theorem \ref{theorem:kmld}.

\begin{theorem}[Koutis \cite{koutis:icalp08} and Williams \cite{williams2009finding}]
\label{theorem:kmld}
There exists an algorithm that, given an instance $P(x_1,\ldots,x_n)$ of the \textsc{$k$-MLD} problem, correctly returns ``no" if $P(X)$ does not contain a $k$ multilinear term. Otherwise, if $P(X)$ has a $k$ multilinear term, it returns ``yes" with probability at least 1/5. 
The algorithm has time complexity $O(2^k poly(n))$ and space complexity $O(2^k poly(n))$.
\end{theorem}

%At a high level, the algorithm of \cite{koutis:icalp08} involves the following steps.
%\begin{enumerate}
%\item For each variable $x_i$, sample a vector $v_i$ uniformly at random from $\mathbb{Z}_2^k$ and assign $x_i = (v_0 + v_i) \in \mathbb{Z}_2[\mathbb{Z}_2^k]$.
%\item Evaluate the polynomial $P(x_1,\ldots,x_n)$ on this random assignment.
%\item If $P(x_1,\ldots,x_n) \neq \bar{0}$ return ``yes"; otherwise, return ``no",
%where $\bar{0}$ is the additive identity of the group algebra.
%\end{enumerate}

%For the case when a multilinear monomial appears an odd number of times in $P(X)$, Koutis proposed a randomized algorithm that runs in $O(2^k poly(n))$ time and returns an affirmative answer with probability at least $1/4$ \cite{koutis:icalp08}. This algorithm was later extended by Williams \cite{williams2009finding} to allow even repetitions.

%It has been shown that many parameterized graph problems reduce to \textsc{$k$-MLD} \cite{koutis:icalp08,williams2009finding,guillemot2013finding,koutis2012constrained} by efficiently encoding subgraphs of interests as monomials. For example, in the $k$-path problem\footnote{In the $k$-path problem, we are given an unweighted graph $G(V,E)$, and the goal is to decide whether or not there is a simple path containing exactly $k$ nodes in $G$.}, it is possible to recursively construct a polynomial in which each term represents a walk of length $k$, and only multilinear terms correspond to simple paths. Our algorithms in Section \ref{sec:proposed} are based on this methodology: we encode subgraphs of a given size and weight as polynomials, and only connected subgraphs of size at most $k$ where all the nodes are different are multilinear terms.



%Now, we define a group algebra $\mathbb{Z}_2[\mathbb{Z}_2^k]$. Each element in the group algebra is a sum of elements from $\mathbb{Z}_2^k$ with coefficients from $\mathbb{Z}_2$ (i.e., either 1 or 0):
%$$
%\sum_{v\in \mathbb{Z}_2^k} a_v v,
%$$
%where $a_v \in \{0,1\}$. Such element may also be interpreted as a subset of $\mathbb{Z}_2^k$---because of the binary coefficients. The addition operator of the group algebra is
%$$
%\sum_{v\in \mathbb{Z}_2^k} a_v v + \sum_{v\in \mathbb{Z}_2^k} b_v v = \sum_{v\in \mathbb{Z}_2^k} (a_v + b_v) v,
%$$
%where the addition of the coefficients is modulo 2. The multiplication is defined as
%$$
%\left(\sum_{v\in \mathbb{Z}_2^k} a_v v\right)\left(\sum_{u\in \mathbb{Z}_2^k} b_u u\right) = \sum_{v\in \mathbb{Z}_2^k} (a_v \cdot b_u) (v\cdot u).
%$$
%

%The key insight in \cite{koutis:icalp08} is that, for any $v_i \in \mathbb{Z}_2^k$,
%{\small
%$$
%(v_0 + v_i)^2 = v_0^2 + 2(v_0\cdot v_i) + v_i^2 = v_0 + (0\mod 2)v_i + v_0 = 2v_0 = 0 \mod 2.
%$$}
%If we are given a polynomial $P(X)$, and we assign uniformly at random an element $(v_0 + v_i)$ from $\mathbb{Z}_2[\mathbb{Z}_2^k]$ to each $x_i$ variable, then, any monomial with a square term will evaluate to 0. This is called the \emph{annihilation} property. The second key idea is that, under this random assignment, a multilinear monomial does not evaluate to 0 with high probability. This is called the \emph{survival} property. Finally, Koutis shows that a polynomial $P(x_1,\ldots,x_n)$, where the variables are elements from $\mathbb{Z}_2[\mathbb{Z}_2^k]$, can be evaluated in time $O(2^k poly(n))$ and space $O(kpoly(n))$.

%Because of the choice of binary coefficients in the group algebra and the modulo 2 addition operation, any monomial that appears an even number of times in $P(X)$ will evaluate to $0$. In \cite{williams2009finding}, Williams proposes working with the group algebra $GF(2^{3 + \log_2k})[\mathbb{Z}_2^k]$, where $GF(p)$ is the finite field of order $p$ \cite{mullen2007finite}. By using this group algebra, it is unlikely that multilinear monomials evaluate to 0 merely due to repetition, and, at the same time, the annihilation property of \cite{koutis:icalp08} is preserved. We have the following theorem.

%\comment{add sequential algorithm in supplementary and refer to it here}

\subsection{Implementation Using a Matrix Representation of $\mathbb{Z}_2[\mathbb{Z}_2^k]$}
Theorem \ref{theorem:kmld} performs operations in the group algebra $\mathbb{Z}_2[\mathbb{Z}_2^k]$,
which takes $O(2^k poly(n))$ space. Koutis \cite{koutis:icalp08} showed that the space complexity can be reduced to $O(k poly(n))$ by using the idea of matrix representations.
%We use the notation from \cite{koutis:icalp08} and describe it here for completeness.
%Let $\rho:\mathbb{Z}_2^k\rightarrow M^{2^k\times 2^k}$ be a matrix representation of
%$\mathbb{Z}_2^k$ satisfying $\rho(uv)=\rho(u)\rho(v)$ for all $u, v\in \mathbb{Z}_2^k$.
%For $k=1$, the map $\rho:\mathbb{Z}_2\rightarrow M^{2\times 2}$ is defined as
%\[
%\rho(0) = 
%\begin{bmatrix}
%1 & 0\\
%0 & 1
%\end{bmatrix}
%\mbox{ and }
%\rho(1) = 
%\begin{bmatrix}
%0 & 1\\
%1 & 0
%\end{bmatrix}
%\]
%
%In general, $\rho(\mathbb{Z}_2^k)$ can be defined recursively. We will use a
%simultaneous diagonalization, which allows $\rho(v)$ to be represented as
%$\rho(v) = U^{-1}\Lambda_v U$, for each $v\in \mathbb{Z}_2^k$, where $\Lambda_v$ is
%a diagonal matrix with its $t$th element equal to $(-1)^{v^T t_{bin}}$,
%where $t_{\text{bin}}$ is the $k$-bit binary representation of $t$.
%Since the algorithm substitutes $v_0+v_i$ for each variable $x_i$, the $t$th element of the diagonal element in the corresponding
%representation for $\rho(v_0+v)$ is $1+(-1)^{v^T t_{bin}}$.
The main idea is that every element of the group algebra can be represented as a $2^{k}\times2^k$ matrix, and the polynomial $P(X)$ evaluates to $\bar 0$ if and only if the trace of its corresponding matrix representation is $0 \mod 2^{k+1}$. We can compute the trace by evaluating the polynomial over the group of all integers $2^k$ times, once for each element of the diagonal. For each variable $x_i = v_0 + v_i$, the $t$th diagonal element in the corresponding matrix representation is $1+(-1)^{v^T t_{bin}}$, where $t_{\text{bin}}$ is the $k$-bit binary representation of $t$.

\subsection{Application to $k$-Path}
As an example of the multilinear detection technique, we now describe a sequential algorithm for the $k$-Path problem, which is a special case of Problem \ref{prob:trees}. We are given a graph $G(V,E)$ and a parameter $k$, and the algorithm decides whether or not the graph has a path of length $k$. First, we reduce $k$-Path to a $k$-MLD instance (this follows from \cite{koutis:icalp08, williams2009finding}). Given a graph $G(V, E)$, let $x_i$ denote a variable associated with each node $i\in V$.
We define poynomials $P(i, j)$ for all $i\in V$, $j\leq k$ in the following manner.

\begin{itemize}
\item
$P(i, 1) = x_i$ for all $i\in V$
\item
For $j>1$,
$P(i, j) = \sum_{u\in\nbr(i)} P(i,1)P(u, j-1)$
\item
Define the polynomial $P(x_1,\ldots,x_n) = \sum_i P(i, k)$
\end{itemize}

Intuitively, a polynomial $P(i, j)$ encodes all the possible walks of length $j$ ending at node $i$. Each monomial in $P(i,j)$ corresponds to one walk. It can be verified that the graph $G$ has a path of length $k$ if and only if the polynomial $P(x_1,\ldots,x_n)$ has a multilinear term---i.e., a walk with no repeated vertices. 

Algorithm \ref{alg:maxwt} presents the full procedure. With the matrix representation, the polynomial for $k$-Path is evaluated over $2^k$ iterations (lines 6---12). In each iteration, we first initialize $P(i, 1)$
%which represents a monomial of degree 1 corresponding to variable $x_i$
(lines 7--8). From there, we compute recursively $P(i,j)$, a polynomial where each term contains $x_i$ and has degree $j$ (lines 9--11). The computation of $P(i,j)$ for a node $i$ uses data from the immediate neighbors of $i$ and all the polynomials of degree $j-1$, which have already been computed at this point. The two applications that we consider in Section \ref{sec:applications} have this structure.

\begin{algorithm}{}
\small
\caption{\small \maxwt{}$(G(V, E), k)$.}
\label{alg:maxwt}
\begin{algorithmic}[1]
\STATE \textbf{Input}: Graph $G(V, E)$ and parameter $k$
\STATE\textbf{Output}: ``Yes" if $G$ has a $k$-Path, ``No" otherwise.
\STATE
\STATE For each node $i$, pick a random vector $v_i \in \mathbb{Z}_{2}^k$
\STATE $P = 0$
\STATE \textbf{for} $t = 0$ to $2^{k-1}$
\STATE \quad $\blacktriangleright$ \textbf{Base case}
\STATE \quad \textbf{for} $i \in V$ \textbf{do}
\STATE \quad \quad $P(i, 1) = 1 + (-1)^{v_i^T \cdot t_{\text{bin}}}$
\STATE \quad $\blacktriangleright$ \textbf{Inductive step}
\STATE \quad \textbf{for} $i \in V$, $j = 2$ to $k$ \textbf{do}
\STATE \quad \quad
$P(i,j) = \sum_{u \in \nbr(i)} P(i, 1)P(u, j-1)$
\STATE \quad $P(k) = \sum_i P(i,k)$ for $i \in V$
\STATE \quad $P = P + P(k)  \mod 2^{k+1}$
\STATE \textbf{return} ``Yes" if $P \neq 0$, else ``No"
\end{algorithmic}
\end{algorithm}

