% !TEX root = ./multilinear.tex
\documentclass[sigconf]{acmart}
\usepackage{booktabs} % For formal tables
\usepackage{algorithm}
\usepackage{algorithmic}
\usepackage[printonlyused,withpage]{acronym}
\usepackage{mdframed}
\usepackage{mdwlist}
\usepackage{enumitem}
\usepackage{wrapfig}
\usepackage{caption}
\usepackage{subcaption}

\usepackage{etoolbox}
\newtoggle{refsfull}
\usepackage{flushend}
\usepackage{multibib}
\newcites{appendix}{References}

%\toggletrue{refsfull} % comment this if you want to hide the references in the short version

%test
%\usepackage{titlesec}
%\titlespacing\section{0pt}{6pt plus 4pt minus 2pt}{0pt plus 2pt minus 2pt}
%\titlespacing\subsection{0pt}{2pt plus 4pt minus 2pt}{1em plus 2pt minus 2pt}
%\titlespacing\subsubsection{0pt}{0pt plus 4pt minus 2pt}{1em plus 2pt minus 2pt}
%\titlespacing\paragraph{0pt}{0pt plus 4pt minus 2pt}{1em plus 2pt minus 2pt}
%\titleformat{\subsection}[runin]{\normalfont\bfseries\large}{}{0 pt}{}[.]
%\titleformat{\subsubsection}[runin]{\normalfont\bfseries}{}{0 pt}{}[.]
%\titleformat{\paragraph}[runin]{\normalfont\itshape}{}{0 pt}{}[.]

\newtheorem{problem}{Problem}
%\newtheorem{theorem}{Theorem}
%\newtheorem{lemma}{Lemma}
\newcommand{\lemmachar}{{\unskip\nobreak\hfil\penalty50\hskip1em\hbox{}%
\nobreak\hfil\rule{1.2ex}{1.4ex}\hfil%
\parfillskip=0pt \finalhyphendemerits=0 \par}}
%\newenvironment{proof}{{\bf Proof:}}{\lemmachar\par}
\newcommand\mc[1]{\multicolumn{1}{c}{#1}} % handy shortcut macro
\def\comment#1{{\color{red}[#1]}}
\makeatletter
%\renewcommand*{\@opargbegintheorem}[3]{\trivlist
%      \item[\hskip \labelsep{\bfseries #1\ #2}] \textbf{(#3)}\ \itshape}
%\makeatother
%\renewcommand{\arraystretch}{1.3}
\newcommand{\cmt}[1]{{\color{red}{#1}}}
\newcommand{\maxwt}{\textsc{MultilinearDetect}}
\newcommand{\parmaxwt}{\textsc{ParMultilinearDetect}}
\newcommand{\mastercompute}{\textsc{MCompute}}
\newcommand{\vertexcompute}{\textsc{VCompute}}
\newcommand{\algcircuit}{\textsc{EvaluateCircuit}}
\newcommand{\parcircuit}{\textsc{ParEvaluateCircuit}}
\newcommand{\algpregel}{\textsc{EvaluateNode}}
\newcommand{\pred}{\textsc{pred}}
\renewcommand{\succ}{\textsc{succ}}
\newcommand{\val}{\textsc{val}}
\newcommand{\op}{\textsc{op}}
\newcommand{\maxload}{\textsc{maxload}}
\newcommand{\maxdeg}{\textsc{maxdeg}}
\newcommand{\nbr}{\textsc{Nbr}}
\newcommand{\myroot}{\textsc{root}}

% Copyright
%\setcopyright{none}
%\setcopyright{acmcopyright}
%\setcopyright{acmlicensed}
%\setcopyright{rightsretained}
%\setcopyright{usgov}
%\setcopyright{usgovmixed}
%\setcopyright{cagov}
%\setcopyright{cagovmixed}


% DOI
\acmDOI{10.475/123_4}

% ISBN
\acmISBN{123-4567-24-567/08/06}

%Conference
\acmConference[SC'17]{}{November 2017}{Denver, CO, USA} 
\acmYear{2017}
\copyrightyear{2017}

\acmPrice{15.00}


\begin{document}
%\title{Constrained Subgraph Maximization in Spark}
\title{Parallel Multilinear Detection and Applications}
%\titlenote{Produces the permission block, and copyright information}
%\subtitle{Extended Abstract}
%\subtitlenote{The full version of the author's guide is available as \texttt{acmart.pdf} document}


%\author{Jose Cadena}
%\affiliation{%
%  \institution{Biocomplexity Institute and \\
%  Virginia Tech}
%  %\streetaddress{P.O. Box 1212}
%  %\city{Dublin} 
%  %\state{Ohio} 
%  %\postcode{43017-6221}
%}
%\email{jcadena@vbi.vt.edu}
%
%%\author{Feng Chen}
%%\affiliation{%
%%  \institution{Dept. of Computer Science\\
%%  State University of New York at Albany}
%%}
%%\email{fchen5@albany.edu}
%
%
%\author{Saliya Ekanayake}
%\affiliation{%
%  \institution{Biocomplexity Institute\\
%  Virginia Tech}
%}
%\email{esaliya@vbi.vt.edu}
%
%\author{Anil Vullikanti}
%\affiliation{\institution{Biocomplexity Institute and \\
%Virginia Tech}}
%\email{akumar@vbi.vt.edu}
%% The default list of authors is too long for headers}
%\renewcommand{\shortauthors}{J. Cadena et. al.}


\begin{abstract}
Different kinds of subgraph detection and analysis problems arise as fundamental
subroutines in graph mining and anomaly detection in a broad class of applications.
Most of the parallel algorithms for such problems are either based on heuristics,
which do not scale very well, or use techniques like color coding, which have a high memory overhead.

In this paper, we develop parallel algorithms using algebraic representations for
such problems---this involves detecting multilinear terms in multivariate polynomials.
We develop parallel algorithms for multilinear detection, and use them for anomaly detection
using graph scan statistics as well as finding specific subgraphs in large networks.
We also compare our MPI based methods with the more recent Pregel based Apache Giraph.
\end{abstract}

%
% The code below should be generated by the tool at
% http://dl.acm.org/ccs.cfm
% Please copy and paste the code instead of the example below. 
%
%\begin{CCSXML}
%<ccs2012>
% <concept>
%  <concept_id>10010520.10010553.10010562</concept_id>
%  <concept_desc>Computer systems organization~Embedded systems</concept_desc>
%  <concept_significance>500</concept_significance>
% </concept>
% <concept>
%  <concept_id>10010520.10010575.10010755</concept_id>
%  <concept_desc>Computer systems organization~Redundancy</concept_desc>
%  <concept_significance>300</concept_significance>
% </concept>
% <concept>
%  <concept_id>10010520.10010553.10010554</concept_id>
%  <concept_desc>Computer systems organization~Robotics</concept_desc>
%  <concept_significance>100</concept_significance>
% </concept>
% <concept>
%  <concept_id>10003033.10003083.10003095</concept_id>
%  <concept_desc>Networks~Network reliability</concept_desc>
%  <concept_significance>100</concept_significance>
% </concept>
%</ccs2012>  
%\end{CCSXML}
%
%\ccsdesc[500]{Computer systems organization~Embedded systems}
%\ccsdesc[300]{Computer systems organization~Redundancy}
%\ccsdesc{Computer systems organization~Robotics}
%\ccsdesc[100]{Networks~Network reliability}
%
%% We no longer use \terms command
%%\terms{Theory}
%
%\keywords{ACM proceedings, \LaTeX, text tagging}


\maketitle

% !TEX root = ./multilinear.tex
\section{Introduction}

Many problems in graph mining and social network analysis can be reduced to questions
about different kinds of subgraphs. Two important classes of such problems, which 
are the focus of our paper, are:
(1) \emph{Detecting and counting subgraphs}, such as paths and trees, of a given size $k$---these are used
for characterizing different kinds of networks, especially in biological models
\cite{alon2008biomolecular, huffner2008algorithm}.
(2) \emph{Anomaly detection in network data using graph scan statistics}, which involves finding
connected subgraphs of size $k$, optimizing different kinds of anomaly score functions \cite{Speakman-14,leiserson2015pan,
hansen2016finding, \iftoggle{refsfull}{neil2013scan,} chen2014non}---this
arises in a number of applications, such as social network analysis, 
epidemiology, finance, and bio-surveillance \cite{akoglu2014graph}.

Both problems are computationally very challenging. For instance, exact detection of
paths is NP-hard and the corresponding counting problem is \#P-hard. Development
of parallel algorithms for these problems is an active area of research, and
many parallel algorithms exist for counting simple local subgraphs, such as triangles \cite{arif:cikm13, schmidt2009scalable, aparicio:ispa14, du:mcd09}. 
Finding trees is much harder, and a number of heuristics have been developed. One
of the few techniques that gives rigorous approximation guarantees is
\emph{color coding} \cite{alon2008biomolecular, alon1995color, huffner2008algorithm}.
Parallel adaptations have been developed using MapReduce \cite{zhao2012sahad} and
MPI \cite{slota:icpp13, slota:ipdps14}. The MPI-based 
FASCIA algorithm \cite{slota:icpp13, slota:ipdps14} is the state-of-the-art in terms of 
counting trees in massive networks, scaling to finding trees with up to 12 vertices in networks with
one billion edges. However, it seems very challenging to scale the color coding
method to larger subgraphs, even on small networks. The main reason is that the
time and space complexity of the color coding technique 
both scale as $2^k$, where $k$ denotes the subgraph size.
In this paper, we take the first steps towards beating this bound,
which has remained a significant open problem since \cite{slota:ipdps14}.
Our approach involves parallelization of a powerful algebraic technique for
detecting multilinear terms in a multivariate polynomial, developed by
Koutis \cite{koutis:icalp08} and Williams \cite{williams2009finding}.

Optimizing network scan statistics leads to challenging optimization problems as well.
Color coding has also been used to develop the first method with rigorous approximation
guarantees \cite{cadena:sdm17}; however, this has not been parallelized because of
its high memory overhead. In \cite{cadena:bigdata17}, we have developed a
parallel adaptation of the multilinear detection technique using both GraphX and Giraph. However,
none of these scaled beyond networks with 40 million edges.

In this paper, we develop a distributed algorithm for multilinear detection,
which immediately leads to highly scalable algorithms for both paths and trees,
and network scan statistics.
Our contributions are:

\noindent
\textbf{1. \ouralgo{}: Distributed algorithm for multilinear detection
and applications to subgraph analysis.}
We develop \ouralgo{}, a randomized distributed MPI based algorithm for finding
paths and trees through detection of
multilinear terms with $k$ variables of the form $x_{i_1}x_{i_2}\ldots x_{i_k}$ (i.e., a term in
which all variables have exponent 1) in a multivariate polynomial
$P(x_1,\ldots,x_n)$. The sequential algorithm uses a matrix representation
of a group algebra \cite{koutis:icalp08, williams2009finding}, and its structure lends itself to a natural
parallelization.  We give rigorous bounds on the performance in terms of the time and
space complexity, which scale as
$O(2^k)$ and $O(k)$, respectively, compared with $O(2^ke^k)$ and $O(2^k)$ for
color coding, respectively \cite{zhao2012sahad, slota:icpp13, slota:ipdps14}.
For random graphs, we show a rigorous scaling with $N$, the number of processors for $N\leq 2^k$.

\noindent
\textbf{2. Cache optimization and weak scalability}. %%Parallelization of the \textsc{$k$-MLD} problem {\color{red} TODO}
Our algorithm partitions the graph $G$ into $N_1$ parts. The computation involves $2^k$
iterations, and $N_1$ processors together perform one iteration---this allows us to schedule
$N/N_1$ such computations to occur in parallel. The total compute time exhibits good weak scaling.
Additionally, our data structures for supporting Galois field operations
during the iterations support a temporal cache locality, which actually leads to a reduction
in the compute time as $N_1$ increases. On the other hand, the communication cost increases
with $N_1$, leading to an optimal value of $N_1$ for the best performance.


\noindent
\textbf{3. Experimental results}. We evaluate our results on a number of real and synthetic networks with up to 250 million edges. The reduced memory footprint allows us to scale to finding subgraphs of size 18, which has not been done before.  Our algorithms for
both problems show reasonable scaling up to 512 processors, supporting our theoretical analysis. 
The running time grows linearly with the network size and as $2^k$ for the subgraph size $k$.

\noindent
\textbf{4. Comparison with prior methods.}
Our algorithm for finding paths is two orders of magnitude faster than FASCIA, the state of the art method based on color coding \cite{slota:icpp13, slota:ipdps14}.
Our algorithm for scan statistics improves on a Giraph-based implementation \cite{cadena:bigdata17} by over one order of magnitude, and it scales to significantly larger networks.

One additional advantage is that our parallel algorithm based on multilinear detection is conceptually much simpler than color coding based algorithms. It also requires far less bookkeeping than color coding. As we discuss in Section \ref{sec:proposed}, the obvious ways to try to parallelize multilinear detection do not scale well; instead, we find that a careful interplay between the degree of partitioning as well as batching a set of iterations yields better results.

%%%%%%%%%%%%%%%%%%%%%%%%%%%%%%%%%%%%%

%%Many problems in graph mining and social network analysis can be reduced to questions
%%about different kinds of subgraphs; two important classes of such problems, which 
%%are the focus of our paper, are:
%%(1) \emph{Detecting subgraphs}, such as paths and trees, of a given size $k$---these are used
%%for characterizing different kinds of networks, especially in biological models
%%\cite{alon2008biomolecular, huffner2008algorithm}.
%%(2) \emph{Anomaly detection in networked data using graph scan statistics}, which involves finding
%%connected subgraphs of size $k$, optimizing different kinds of objectives \cite{Speakman-14,leiserson2015pan,
%%hansen2016finding, \iftoggle{refsfull}{neil2013scan,} chen2014non}.
%%This problem arises in a number of applications, such as social network analysis, 
%%epidemiology, finance, and bio-surveillance. We refer to the survey by 
%%Akoglu et al. \cite{akoglu2014graph} for a detailed discussion on different approaches 
%%to graph anomaly detection.
%%
%%The problems of subgraph detection and scan statistics are NP-hard, in general, making them 
%%computationally very challenging. Various sequential heuristics have been proposed for
%%both these problem classes, e.g.,
%%\cite{alon2008biomolecular, huffner2008algorithm, Speakman-14,leiserson2015pan,
%%hansen2016finding, \iftoggle{refsfull}{neil2013scan,} chen2014non, cheng:kdd12}. However, these
%%do not scale to large instances, motivating the need for parallel implementations.
%%There has been a fair amount of work on parallel algorithms for various kinds of subgraph
%%problems, e.g., \cite{zhao2012sahad, slota:icpp13, slota:ipdps14, arif:cikm13, 
%%schmidt2009scalable, zhao2016parallel, aparicio:ispa14, du:mcd09}, but very little for
%%anomaly detection using graph scan statistics \cite{zhao2016parallel}.
%%
%%
%%Most of these parallel algorithms involve heuristics based on partitioning, pruning, and 
%%enumeration for speeding up the computations to fairly large graphs---e.g., for 
%%maximal cliques \cite{schmidt2009scalable, zhao2016parallel, aparicio:ispa14, cheng:kdd12, du:mcd09}.
%%One of the few rigorous approaches for subgraph counting that has been used for
%%developing parallel algorithms is based on the ``color coding'' technique
%%\cite{alon2008biomolecular, alon1995color}, which gives a rigorous \emph{fixed parameter tractable} algorithm\footnote{This means the running time is $O(f(k)n^c)$, where
%%$n$ is the number of nodes, $c$ is a constant, and $f(\cdot)$ is a function of some other
%%parameter---in this case, it is the size of the subgraph \cite{downey}.}
%%for finding and counting subgraphs. Color coding is a dynamic programming algorithm,
%%and multiple approaches have been developed for parallelizing it 
%%\cite{zhao2012sahad, slota:icpp13, slota:ipdps14}; the FASCIA algorithm of \cite{slota:icpp13, slota:ipdps14}
%%which is implemented in MPI is the state-of-the-art, and can handle trees
%%of size up to 12 in graphs with millions of nodes. 
%%%%Color coding has also been used to develop a
%%%%sequential algorithm for graph scan statistics \cite{cadena:sdm17}, but no parallel
%%%%adaptations exist so far.
%%A big challenge of the color coding technique is the memory overhead, since
%%it scales as $2^k$, where $k$ denotes the subgraph size, which limits further
%%scaling to larger graphs or subgraphs.  
%%
%%In this paper, we develop a novel approach for designing highly scalable parallel
%%algorithms for both the problem classes mentioned above, by adapting
%%powerful algebraic techniques for
%%detecting multilinear terms in a multivariate polynomial developed by 
%%Koutis \cite{koutis:icalp08} and Williams \cite{williams2009finding}.
%%Our contributions are:
%%
%%\noindent
%%\textbf{1. Efficient algorithms for parallel multilinear detection
%%and applications to subgraph analysis.}
%%We develop a distributed MPI based algorithm for detecting 
%%multilinear terms with $k$ variables of the form $x_{i_1}x_{i_2}\ldots x_{i_k}$ (i.e., a term in
%%which all variables have exponent 1) in a multivariate polynomial
%%$P(x_1,\ldots,x_n)$---this is referred to as the $k$-Multilinear Detection (\textsc{$k$-MLD}) problem
%%\cite{koutis:icalp08, williams2009finding}. The problem of finding paths and trees can
%%be reduced to the \textsc{$k$-MLD} problem \cite{koutis:icalp08}. We also show that the
%%network scan statistics problem can also be reduced to the \textsc{$k$-MLD} problem.
%%Consequently, our parallel multilinear detection algorithm leads to parallel algorithms
%%for both these problems. Our methods provably
%%have better space and running time complexity compared to the color coding technique.
%%The space complexity is $O(kn)$ compared to $O(2^kn)$, whereas the running time complexity is proportional to $O(2^k)$, compared to $O(2^ke^k)$
%%\cite{zhao2012sahad, slota:icpp13, slota:ipdps14}.
%%
%%\noindent
%%\textbf{2. Experimental results}. We evaluate our results on a number of real and synthetic networks 
%%with up to 250 million edges and subgraph sizes up to $k=18$. The reduced memory footprint
%%allows us to scale to paths of size 18, which has not been done before.  Our algorithms for
%%both problems show reasonable scaling up to 512 processors. The running time grows linearly
%%with the network size and as $2^k$ for the subgraph size $k$.
%%
%%\noindent
%%\textbf{3. Comparison with prior methods.}
%%Our algorithm for finding paths gives over two orders of magnitude improvement in time
%%compared to FASCIA, the state of the art method based on color coding \cite{slota:icpp13, slota:ipdps14}.
%%There is only one parallel algorithm for network scan statistics, which is based on pruning heuristics,
%%and our method gives several order of magnitude improvement over it.
%%
%%\noindent
%%\textbf{4. Comparison with Giraph.}
%%We compare our methods with recent parallel graph programming models, Giraph and Spark.
%%Our results show that neither of these models scale well for these problems; further,
%%Spark has worse performance than Giraph.


%we study a novel approach for a broad class of subgraph analysis problems
%in massive networks based on powerful algebraic techniques. This involves formalizing these
%problems in terms of detecting multilinear terms in a multivariate polynomial, based on 
%very powerful techniques developed by Koutis \cite{koutis:icalp08} and Williams \cite{williams2009finding}.
%However, only sequential algorithms have been developed so far \cite{bjorklund:esa14}, which do not
%scale beyond networks with $\sim 10^3-10^4$ nodes.
%We develop a novel approach for parallel multilinear detection, which allows us to
%solve a broad class of subgraph analysis and anomaly detection problems with the same tools.
%
%Both multilinear detection and subgraph counting using color coding are examples of
%\emph{fixed parameter tractable} algorithms, which run in time $O(f(k)n^c)$, where
%$n$ is the number of nodes, $c$ is a constant, and $f(\cdot)$ is a function of some other
%parameter---in this case, it is the size of the subgraph. Most of these problems are
%NP-complete, in general, and such algorithms are one way to handle the hardness; this is
%referred to as \emph{parameterized complexity} (see \cite{downey} for an excellent discussion
%of such methods for many fundamental NP-hard problems).
%
%In this paper, we study parallel algorithms for multilinear detection and their applications
%to different subgraph analysis problems, using multiple models of parallel graph computation.
%Our contributions are:


% !TEX root = ./multilinear.tex
\section{Preliminaries}
\label{sec:prelim}

\subsection{The Multilinear Detection Problem}
Let $X = x_1, \ldots,x_n$ be a set of variables, and let $P(X)$ be a polynomial, which is a sum 
of monomials on $X$. We will denote $P(X)=\sum_S \Pi_{i\in S} x_i$ as a monomial, where
the sum is over multisets $S$.  An example of a polynomial on six variables is 
$P(x_1,x_2,x_3,x_4, x_5, x_6) = x_1^2x_2 + x_2x_3x_4 + x_3x_4x_5 + x_5x_6$. 
A monomial is called \emph{multilinear} or \emph{square-free} if all its variables 
have exponent 1, and its \emph{degree} is the sum of the exponents of all its variables. 
For instance, in the example above, $x_2x_3x_4$, $x_3x_4x_5$, and $x_5x_6$ are multilinear monomials, but $x_1^2x_2$ is not multilinear. 
Given variables $X = x_1, \ldots ,x_n$ and a polynomial $P(X)$, the goal in 
the $k$-Multilinear Detection (\textsc{$k$-MLD}) problem
is to decide whether or not $P(X)$ has a multilinear monomial of degree exactly $k$. 


We note that the polynomial $P(X)$ may have an arbitrary number of terms---i.e., exponential on the size of $n$---therefore, the problem is not as simple as writing the polynomial explicitely and checking each term. Rather, we assume that $P(X)$ is given succintly in a recursive form, and the ``yes"/``no" decision has to be made without unrolling this recursion.
Figure \ref{fig:dag} illustrates a circuit representation of the polynomial $P(\cdot)$ above.
In general, we also have a weight $w_S$ for each multinomial $\Pi_{i\in S} x_i$ corresponding
to each term in the polynomial $P(\cdot)$. Our focus in this paper will be the following problem.

\begin{figure}[h]
\includegraphics[width=0.5\textwidth]{img/dag3_fixed.pdf}
\caption{
\small
The polynomial $P(x_1,x_2,x_3,x_4, x_5, x_6) = x_1^2x_2 + x_2x_3x_4 + x_3x_4x_5 + x_5x_6$
represented as a circuit with multiplication ($\cdot$) and addition ($+$) gates, which is a DAG. 
Each variable $x_i$ is assigned the element $v_0+v_i\in\mathbb{Z}_2[\mathbb{Z}_2^2]$,
for randomly chosen $v_i\in \mathbb{Z}_2^2$. The computations at each gate
are in the group algebra, and are shown in blue. The circuit evaluates to the element $J$.
\vspace{-0.2in}
}
\label{fig:dag}
\end{figure}

\begin{problem} (\textsc{$k$-MLD} problem)
Given a polynomial $P(\cdot)$ represented by a DAG $G=(V, E)$, in which each monomial
has degree at most $k$, with a weight $w_S$ for
each monomial corresponding to the multiset $S$ in $P(\cdot)$, determine:
(1) if $P(\cdot)$ has a multilinear term, and 
(2) the maximum weight of any multilinear term, if one exists.
\end{problem}


\subsection{Group Algebras}
\label{sec:grpalgebra}
We discuss some notation from group algebras that is crucial for the paper. 
Let $\mathbb{Z}_2^k$ be the group formed by all the $k$-dimensional binary vectors, and define the group multiplication operation as entry-wise XOR. For example, $\mathbb{Z}_2^2$ consists of the vectors $v_0 = (0, 0), v_1 = (0, 1), v_2 = (1, 0), v_3 = (1, 1)$. We note that $v_0$ is the multiplicative identity of the group, and each element is its own multiplicative inverse: $v_i \cdot v_i = v_0$. Now, we define a group algebra $\mathbb{Z}_2[\mathbb{Z}_2^k]$. Each element in the group algebra is a sum of elements from $\mathbb{Z}_2^k$ with coefficients from $\mathbb{Z}_2$ (i.e., either 1 or 0):
$
\sum_{v\in \mathbb{Z}_2^k} a_v v,
$
where $a_v \in \{0,1\}$. The addition operator of the group algebra is
{\scriptsize
$$
\sum_{v\in \mathbb{Z}_2^k} a_v v + \sum_{v\in \mathbb{Z}_2^k} b_v v = \sum_{v\in \mathbb{Z}_2^k} (a_v + b_v) v,
$$}
where the addition of the coefficients is modulo 2, and the multiplication is defined as
{\scriptsize
$$
\left(\sum_{v\in \mathbb{Z}_2^k} a_v v\right)\left(\sum_{u\in \mathbb{Z}_2^k} b_u u\right) = \sum_{v\in \mathbb{Z}_2^k} (a_v \cdot b_u) (v\cdot u).
$$}


\begin{mdframed}
\scriptsize{
\noindent
\textbf{Example.} For $k=2$, the group algebra $\mathbb{Z}_2[\mathbb{Z}_2^k]$ has
$2^{2^2}=16$ elements, such as
\[
x_1=
0\cdot
\begin{bmatrix}
0\\
0
\end{bmatrix}
+ 1\cdot
\begin{bmatrix}
0\\
1
\end{bmatrix}
+
1\cdot
\begin{bmatrix}
1\\
0
\end{bmatrix}
+ 0\cdot
\begin{bmatrix}
1\\
1
\end{bmatrix}
, \mbox{ which we also write as }
\begin{bmatrix}
0\\
1
\end{bmatrix}
+
\begin{bmatrix}
1\\
0
\end{bmatrix}
\]


\[
x_2 =
\begin{bmatrix}
0\\
0
\end{bmatrix}
+
\begin{bmatrix}
1\\
0
\end{bmatrix}
\]

We have 
\[
x_1 + x_2 = 
\begin{bmatrix}
0\\
0
\end{bmatrix}
+
\begin{bmatrix}
0\\
1
\end{bmatrix}
+ 2
\begin{bmatrix}
1\\
0
\end{bmatrix}
=
\begin{bmatrix}
0\\
0
\end{bmatrix}
+
\begin{bmatrix}
0\\
1
\end{bmatrix}
\]

\[
x_1x_2 = 
\left(
\begin{bmatrix}
0\\
1
\end{bmatrix}
+ 
\begin{bmatrix}
1\\
0
\end{bmatrix}
\right)\cdot 
\left(
\begin{bmatrix}
0\\
0
\end{bmatrix}
+
\begin{bmatrix}
1\\
0
\end{bmatrix}
\right) =
\begin{bmatrix}
0\\
0
\end{bmatrix}
+
\begin{bmatrix}
0\\
1
\end{bmatrix}
+
\begin{bmatrix}
1\\
0
\end{bmatrix}
+
\begin{bmatrix}
1\\
1
\end{bmatrix}
\]
It is easy to check that
\[
x_1^2x_2 = 
0\cdot
\begin{bmatrix}
0\\
0
\end{bmatrix}
+ 0\cdot
\begin{bmatrix}
0\\
1
\end{bmatrix}
+ 0\cdot
\begin{bmatrix}
1\\
0
\end{bmatrix}
+ 0\cdot
\begin{bmatrix}
1\\
1
\end{bmatrix}
= \mathbf{\bar{0}} \text{ (additive identify)}
\]
}
\end{mdframed}


\section{Sequential algorithm for Multilinear Detection} 
\label{sec:seq}
We briefly discuss the algorithm of Koutis \cite{koutis:icalp08}, which forms the basis of
our paper. An important property is that for any $v_i \in \mathbb{Z}_2^k$, the square of the term $(v_0 + v_i) \in \mathbb{Z}_2[\mathbb{Z}_2^k]$ evaluates to 0:
{\small
$$
(v_0 + v_i)^2 = v_0^2 + 2(v_0\cdot v_i) + v_i^2 = v_0 + (0\mod 2)v_i + v_0 = 2v_0 = 0.
$$}

The main idea in the algorithm of \cite{koutis:icalp08} is that, if we evaluate a polynomial 
over the ``right" algebra, monomials that have square terms evaluate to 0, and the remaining terms,
which are multilinear, do not cancel out, with some probability.
Then, a polynomial $P(X)$ has a $k$ multilinear term if $P(X) \neq 0$. 

We can show that, if we choose a $v_i \in \mathbb{Z}_2^k$ uniformly at random and set $x_i = v_0 + v_i$, then a multilinear monomial does \textbf{not} evaluate to $\bar 0$ with high probability, whereas a monomial with squares is always $\bar 0$ (as in the box above).
The algorithm was later refined in \cite{williams2009finding} by evaluating the polynomial over the group algebra $GF(2^{3 + \log_2k})[\mathbb{Z}_2^k]$, where $GF(p)$ is the finite field of order $p$ \cite{mullen2007finite}. A polynomial $P(x_1,\ldots,x_n)$ with variables from $GF(2^{3 + \log_2k})[\mathbb{Z}_2^k]$ can be evaluated in time $O(2^k poly(n))$ and space $O(kpoly(n))$, resulting in Theorem \ref{theorem:kmld}.

\begin{theorem}[Koutis \cite{koutis:icalp08} and Williams \cite{williams2009finding}]
\label{theorem:kmld}
There exists an algorithm that, given an instance $P(x_1,\ldots,x_n)$ of the \textsc{$k$-MLD} problem, correctly returns ``no" if $P(X)$ does not contain a $k$ multilinear term. Otherwise, if $P(X)$ has a $k$ multilinear term, it returns ``yes" with probability at least 1/5. 
The algorithm has time complexity $O(2^k poly(n))$ and space complexity $O(2^k poly(n))$.
\end{theorem}

At a high level, the algorithm of \cite{koutis:icalp08} involves the following steps.
\begin{enumerate}
\item For each variable $x_i$, sample a vector $v_i$ uniformly at random from $\mathbb{Z}_2^k$ and assign $x_i = (v_0 + v_i) \in \mathbb{Z}_2[\mathbb{Z}_2^k]$.
\item Evaluate the polynomial $P(x_1,\ldots,x_n)$ on this random assignment.
\item If $P(x_1,\ldots,x_n) \neq \bar{0}$ return ``yes"; otherwise, return ``no",
where $\bar{0}$ is the additive identity of the group algebra.
\end{enumerate}

%For the case when a multilinear monomial appears an odd number of times in $P(X)$, Koutis proposed a randomized algorithm that runs in $O(2^k poly(n))$ time and returns an affirmative answer with probability at least $1/4$ \cite{koutis:icalp08}. This algorithm was later extended by Williams \cite{williams2009finding} to allow even repetitions.

%It has been shown that many parameterized graph problems reduce to \textsc{$k$-MLD} \cite{koutis:icalp08,williams2009finding,guillemot2013finding,koutis2012constrained} by efficiently encoding subgraphs of interests as monomials. For example, in the $k$-path problem\footnote{In the $k$-path problem, we are given an unweighted graph $G(V,E)$, and the goal is to decide whether or not there is a simple path containing exactly $k$ nodes in $G$.}, it is possible to recursively construct a polynomial in which each term represents a walk of length $k$, and only multilinear terms correspond to simple paths. Our algorithms in Section \ref{sec:proposed} are based on this methodology: we encode subgraphs of a given size and weight as polynomials, and only connected subgraphs of size at most $k$ where all the nodes are different are multilinear terms.



%Now, we define a group algebra $\mathbb{Z}_2[\mathbb{Z}_2^k]$. Each element in the group algebra is a sum of elements from $\mathbb{Z}_2^k$ with coefficients from $\mathbb{Z}_2$ (i.e., either 1 or 0):
%$$
%\sum_{v\in \mathbb{Z}_2^k} a_v v,
%$$
%where $a_v \in \{0,1\}$. Such element may also be interpreted as a subset of $\mathbb{Z}_2^k$---because of the binary coefficients. The addition operator of the group algebra is
%$$
%\sum_{v\in \mathbb{Z}_2^k} a_v v + \sum_{v\in \mathbb{Z}_2^k} b_v v = \sum_{v\in \mathbb{Z}_2^k} (a_v + b_v) v,
%$$
%where the addition of the coefficients is modulo 2. The multiplication is defined as
%$$
%\left(\sum_{v\in \mathbb{Z}_2^k} a_v v\right)\left(\sum_{u\in \mathbb{Z}_2^k} b_u u\right) = \sum_{v\in \mathbb{Z}_2^k} (a_v \cdot b_u) (v\cdot u).
%$$
%

%The key insight in \cite{koutis:icalp08} is that, for any $v_i \in \mathbb{Z}_2^k$,
%{\small
%$$
%(v_0 + v_i)^2 = v_0^2 + 2(v_0\cdot v_i) + v_i^2 = v_0 + (0\mod 2)v_i + v_0 = 2v_0 = 0 \mod 2.
%$$}
%If we are given a polynomial $P(X)$, and we assign uniformly at random an element $(v_0 + v_i)$ from $\mathbb{Z}_2[\mathbb{Z}_2^k]$ to each $x_i$ variable, then, any monomial with a square term will evaluate to 0. This is called the \emph{annihilation} property. The second key idea is that, under this random assignment, a multilinear monomial does not evaluate to 0 with high probability. This is called the \emph{survival} property. Finally, Koutis shows that a polynomial $P(x_1,\ldots,x_n)$, where the variables are elements from $\mathbb{Z}_2[\mathbb{Z}_2^k]$, can be evaluated in time $O(2^k poly(n))$ and space $O(kpoly(n))$.

%Because of the choice of binary coefficients in the group algebra and the modulo 2 addition operation, any monomial that appears an even number of times in $P(X)$ will evaluate to $0$. In \cite{williams2009finding}, Williams proposes working with the group algebra $GF(2^{3 + \log_2k})[\mathbb{Z}_2^k]$, where $GF(p)$ is the finite field of order $p$ \cite{mullen2007finite}. By using this group algebra, it is unlikely that multilinear monomials evaluate to 0 merely due to repetition, and, at the same time, the annihilation property of \cite{koutis:icalp08} is preserved. We have the following theorem.


\subsection{Implementing The Sequential Algorithm Using a Matrix Representation
of $\mathbb{Z}_2[\mathbb{Z}_2^k]$}

Theorem \ref{theorem:kmld} performs operations in the group algebra $\mathbb{Z}_2[\mathbb{Z}_2^k]$,
which takes $O(2^k poly(n))$ space. Koutis \cite{koutis:icalp08} showed that, in fact, the space
complexity can be reduced to $O(k poly(n))$ by using the idea of matrix representations.
We describe this approach here and then the sequential algorithm \maxwt{}
for $k$-MLD. We will develop the parallel version of this algorithm in 
Section \ref{sec:proposed}.

We use the notation from \cite{koutis:icalp08} and describe it here for completeness.
Let $\rho:\mathbb{Z}_2^k\rightarrow M^{2^k\times 2^k}$ be a matrix representation of
$\mathbb{Z}_2^k$ satisfying $\rho(uv)=\rho(u)\rho(v)$ for all $u, v\in \mathbb{Z}_2^k$.
For $k=1$, the map $\rho:\mathbb{Z}_2\rightarrow M^{2\times 2}$ is defined as
\[
\rho(0) = 
\begin{bmatrix}
1 & 0\\
0 & 1
\end{bmatrix}
\mbox{ and }
\rho(1) = 
\begin{bmatrix}
0 & 1\\
1 & 0
\end{bmatrix}
\]

In general, $\rho(\mathbb{Z}_2^k)$ can be defined recursively. We will use a
simultaneous diagonalization, which allows $\rho(v)$ to be represented as
$\rho(v) = U^{-1}\Lambda_v U$, for each $v\in \mathbb{Z}_2^k$, where $\Lambda_v$ is
a diagonal matrix with its $t$th element equal to $(-1)^{v^T t_{bin}}$,
where $t_{\text{bin}}$ is the $k$-bit binary representation of $t$.
Since the algorithm substitutes $v_0+v_i$ for each variable $x_i$ (as shown in
Figure \ref{fig:dag}), the $t$th element of the diagonal element in the corresponding
representation for $\rho(v_0+v)$ is $1+(-1)^{v^T t_{bin}}$.

We now describe the sequential algorithm \maxwt{} for the $k$-MLD problem.
We are given a circuit $G(V,E)$ and parameter $k$, which together constitute an instance of $k$-MLD. 
The circuit is a directed acyclic graph (DAG).
For each node $i$ in the circuit, we define ${\textsc{pred}}(i)$ and $\succ(i)$ to be, 
respectively, the set of predecessors and successors of $i$. Nodes without predecessors 
are the \emph{circuit inputs}. Additionally, the circuit has one single node without 
successors, which we denote by root$(G)$; this is the \emph{root} of the circuit. 
Since $G$ is a DAG, it can be partitioned into
sets of levels $\mathcal{L} = \{L_1, L_2, \ldots, L_{|\mathcal{L}|}\}$, where:
\begin{itemize}
\item
$L_1 = \{i \in V: {\textsc{pred}}(i) = \emptyset\}$
is the set of nodes with no predecessors (i.e., the input nodes)
\item
For all $s>1$,
$L_s = \{i \in V | \exists j: j\in {\textsc{pred}}(i) \wedge j \in L_{s-1}\}$
consists of nodes with at least one predecessor in $L_{s-1}$.
\end{itemize}

Each non-input node $i$ in $G$ is associated with an \emph{operation} $\textsc{op}(i) \in \{+, \cdot\}$, 
which corresponds to an addition or a multiplication operation at that node.
We denote as $\textsc{val}(i)$ the result of performing operation $\textsc{op}(i)$ on the predecessors of $i$. 
If $\textsc{op}(i) = +$, then
$$
\val(i) = \sum_{j \in {\pred}(i)} \val(j),
$$
and analogously for $\op(i) = \cdot$. For the circuit inputs, which have no predecessors, 
$\val(i)$ is a value that we assign in the algorithm.

\noindent
\textbf{High level idea of \maxwt{} (Algorithm \ref{alg:multilinear-detect}).}
After selecting random vectors from $\mathbb{Z}_2^k$ for each circuit input (lines 4--5), 
we compute the polynomial by performing $2^k$ evaluations of the circuit (lines 9--10). 
The $t$-{th} evaluation gives us the value of the $(t + 1)$-th row in the matrix 
representation, as discussed in Section \ref{sec:prelim}. The procedure \algcircuit{}, 
performs the evaluation by levels. First, we initialize the circuit inputs (lines 18--19) 
to be the $t$-th eigenvalue of their respective matrix representation, which is computed 
as $1 + (-1)^{v_i^T\cdot t_{\text{bin}}}$, where $v_i$ is the random vector assigned to node 
$i$ and $t_{\text{bin}}$ is the $k$-bit binary representation of $t$. 
Then, for each level $L_s$ and node $i\in L_s$, we obtain $\val(i)$ by aggregating 
values from nodes in levels $L_{s-1}$ or below, which have already been computed in 
a previous iteration of the \textbf{for} loop (lines 20--30). 
We show an example of this algorithm in Figure \ref{fig:algorithm-example}.

\begin{figure*}[!htbp]
\includegraphics[width=\textwidth]{img/algorithm-example-v2_fixed.pdf}
\caption{
\small
Example of the sequential algorithm \maxwt{}. The input is the circuit $G(V,E)$ from
Figure \ref{fig:dag} with 12 nodes on four levels, $L_1$, $L_2$, $L_3$ and $L_4$, as shown.
The algorithm picks the vectors $v_1,\ldots,v_6$ from $\mathbb{Z}_2^2$, as shown in
the left of the figure. Since $k=2$, there are $2^k=4$ iterations, corresponding to
$t=0,\ldots,3$. The values of $\val(v)=1+(-1)^{v^Tt_{bin}}$ for the $t$th iteration
are shown in blue as the input values. The outputs of the circuit are $M_0, \ldots, M_3$,
with values as shown in the right of the figure. For these choices of the $v_i$'s, the
computed value $M\neq 0 (\text{mod } 2^{k+1})$, which implies the existence of a
multilinear term with $k$ variables. This is true because the term $x_5x_6$ in the
polynomial is indeed multilinear.
%---IDs are in red---$k=2$, and 3 levels. Nodes 1 through 6 are the first level---i.e., the circuit inputs. Nodes 7 through 10 are the second level, all with multiplication operation ($op(i) = \cdot$). The last level only contains the root of the circuit, with addition operation. First, the algorithm generates random vectors $v_i$ for the circuit inputs (bottom left). Then, the circuit is evaluated $2^k$ times by calling procedure \algcircuit{} (center). We show the evaluation for iteration $t=2$. First, we compute the inputs to the circuit as $1 + (-1)^{v_i^T\cdot t_{\text{bin}}}$---this will always be either 2 or 0. From there, we can evaluate the circuit by levels. Finally, back in \maxwt{}, we aggregate the value at root$(G)$ over all the iterations (right). In this case, the polynomial evaluates to $0$, so we return ``False".
%\vspace{-0.2in}
}
\label{fig:algorithm-example}
\end{figure*}

\begin{algorithm}{}
\small
\caption{\maxwt{}$(G(V, E), k, \mathcal{L})$.}
\label{alg:multilinear-detect}
\begin{algorithmic}[1]
\STATE \textbf{Input}: Circuit $G(V, E)$, parameter $k$, and node levels $\mathcal{L}$
\STATE\textbf{Output}: ``True" if circuit evaluation is non-zero. ``False" otherwise.
\STATE \textbf{Initialize circuit inputs}
\STATE \textbf{for} node $i \in L_1$ \textbf{do}
\STATE \quad Let $v_i$ be a random vector from $\mathbb{Z}_{2}^k$
\STATE \textbf{Initialize the polynomial}
\STATE Let $M = \bar 0$
\STATE \textbf{Evaluate circuit for each row of matrix representation}
\STATE \textbf{for} $t = 0$ to $2^{k-1}$ \textbf{do}
\STATE \quad $M_t = \algcircuit(G(V, E), k, \mathcal{L}, \mathbf{v}, t)$
\STATE $M = \sum_{t=0}^{2^{k-1}} M_t \mod 2^{k+1}$
\STATE \textbf{return} $M \neq 0$
\STATE
\STATE \textbf{procedure} \algcircuit{$(G(V, E), k, \mathcal{L}, \mathbf{v}, t)$}
\STATE \textbf{Input}: Circuit $G(V, E)$, parameter $k$, node levels $\mathcal{L}$, random assigment $\mathbf{v}$, and iteration number $t$
\STATE\textbf{Output}: Value at root node of $G(V,E)$
\STATE \textbf{Initialize circuit inputs}
\STATE \textbf{for} node $i \in L_1$ \textbf{do}
\STATE \quad $ \val(i) = 1 + (-1)^{v_i^T \cdot t_{\text{bin}}}$

\STATE \textbf{Evaluate the circuit by levels}
\STATE \textbf{for} $s=2$ to $|\mathcal{L}|$ \textbf{do}
\STATE \quad \textbf{for} $i \in L_s$ \textbf{do}
\STATE \qquad \textbf{if} $\op(i) = +$ \textbf{then}
\STATE \qquad \quad $\val(i) = 0$
\STATE \qquad \quad \textbf{for} $j \in \pred(i)$ \textbf{do}
\STATE \qquad \qquad $\val(i) = \val(i) + \val(j)$
\STATE \qquad \textbf{else} 
\STATE \qquad \quad $\val(i) = 1$
\STATE \qquad \quad \textbf{for} $j \in \pred(i)$ \textbf{do}
\STATE \qquad \qquad $\val(i) = \val(i) \cdot \val(j)$

\STATE \textbf{return} $\val($root$(G))$
\end{algorithmic}
\end{algorithm}

\begin{theorem}[Koutis \cite{koutis:icalp08} and Williams \cite{williams2009finding}]
\label{theorem:kmld2}
Algorithm \maxwt{} correctly solves the \textsc{$k$-MLD} problem for an 
instance $P(x_1,\ldots,x_n)$ with probability at least 1/5 in time
$O(2^k poly(n))$ and space $O(2^k n)$.
\end{theorem}

%We note that the success probability in Theorem \ref{theorem:kmld2} can be made very high,
%e.g., $1-\frac{1}{n}$, by running the algorithm $O(\log{n})$ times.

% !TEX root = ./multilinear.tex

\section{Parallel algorithm: abstract description using a DAG representation}
\label{sec:proposed}

We describe our parallel algorithms in a more general form, for finding multilinear
terms in a multivariate polynomial, which can be represented as a DAG. 
Both problems \ref{prob:trees} and \ref{prob:macs} can be reduced to multilinear detection---this
follows from \cite{koutis:icalp08,williams2009finding} for problem \ref{prob:trees} and
from Section \ref{sec:scan-sequential} for problem \ref{prob:macs}. 
We note that the DAG representation
is only for the purpose of explaining the parallel algorithm for both these problems; we do not
solve the two problems by explicitly constructing DAGs. We will describe the specific algorithms
for both the problems by constructing the DAGs implcitly in Section \ref{sec:application}.

\subsection{Overview of the Algorithm}
We assume a DAG $G(V,E)$ which represents an instance of $k$-MLD, with levels
$\mathcal{L} = \{L_1,\ldots,L_{|\mathcal{L}|}\}$. The algorithm involves evaluating the
DAG in $2^k$ iterations, which are partitioned into ``phases'' of $N_2$ iterations each.
The idea is that a phase of $N_2$ iterations will be done simultaneously using $N_1$
processors. Let $N$ denote the total number of processors available. Therefore, $N/N_1$
phases can be done in parallel.

Let $N$ denote the total number
of processors available; of these $N_1$ will be used for each iteration. 
We assume the graph is partitioned into $N_1\leq N$ parts, denoted by
$\mathcal{P}=\{G^1= (V^1, E^1), \ldots, G^{N_1}=(V^{N_1}, E^{N_1})\}$; desirable properties of
the partition will be discussed later.
For each level $L_s\in\mathcal{L}$, let $L_s^j$ denote the corresponding level
in partition $j$; $\mathcal{L}^j$ denotes the set of all levels in that partition.
For level $L_s\in\mathcal{L}$, let $\maxload{}_s = \max_j |L_s^j|$ be the maximum
``load'' on partition $j$ from that level. Further, for $s<|\mathcal{L}|$, let
$\maxdeg{}_s = \max_{j\neq j'} |\{(u, v)\in L_s\times L_{s+1}: u\in L_s^j, v\in L_{s+1}^{j'}\}|$
denote the maximum number of edges from any $L_s^j$ to $L_{s+1}^{j'}$.
We will analyze the performance of our algorithm in terms of $\maxload{}_s$ 
and $\maxdeg{}_s$,
which motivates the properties of the partitioning to ensure best performance.

\begin{figure}[h]
\includegraphics[width=0.4\textwidth]{img/dag4.pdf}
\caption{
\small
Illustration of the level sets and associated quantities for the DAG in
Figure \ref{fig:dag}, corresponding to a partition into two parts. 
%\vspace{-0.2in}
}
\label{fig:dag4}
\end{figure}

We describe the main intuition of the steps of Algorithm \parmaxwt{} below:
\begin{enumerate}
\item
The algorithm starts with the partitioning $\mathcal{P}$ of the graph $G$.
We discuss its complexity and implementation in Section \ref{sec:partition}.
\item
Each of the $2^k$ iterations in the while loop in lines 10-15 of \maxwt{}
is completely independent of others. Since $N_1$ processors are used for each iteration,
we run $\lfloor{N/N_1}\rfloor$ iterations in parallel, which require $\frac{2^k}{\lfloor{N/N_1}\rfloor}$
iterations of the while loop.
\item
Algorithm \parcircuit{} evaluates the circuit by levels, and computes
a value $\val(i)$ for each node $i$.
\item
For each node $i$ that we evaluate, we compute $\val(i)$ using the $\val(j)$ value for each predecessor $j$. 
If a predecessor $j$ is in the same partition $p'$, then, we can simply read $\val(j)$, 
as in the sequential algorithm. For every predecessor $j$ in a different partition, 
$j$ has to send a message with $\val(j)$, introducing a communication overhead.
\item
We aggregate the value at root$(G)$ over all $2^{k}$ iterations to get the final solution.
\end{enumerate}


\begin{algorithm}{}
\small
\caption{\parmaxwt{}$(G(V,E), k, \mathcal{L}, \epsilon, N_1, N_2)$.}
\label{alg:parallel-kMLD} 
\begin{algorithmic}[1]
\STATE \textbf{Input}: Circuit $G(V, E)$, parameter $k$, node levels $\mathcal{L}$,
confidence parameter $\epsilon\in (0, 1)$, parameters $N_1$ and $N_2$, which guide the parallelism.
\STATE\textbf{Output}: ``True" if circuit evaluation is non-zero. ``False" otherwise.

\STATE Let $v_i \in \mathbb{Z}_2^k$ be a random vector for each node $i$
\STATE Let $M = \bar 0$ be the polynomial
\STATE Let $N_1$ denote the number of processors used for each iteration.
Let $\mathcal{P}=\{G^1= (V^1, E^1), \ldots, G^{N_1}=(V^{N_1}, E^{N_1})\}$ denote the corresponding
partition of the graph into $N_1$ parts.
\STATE Let $a = \lfloor N / N_1 \rfloor$ be the number of iterations that will be run in parallel
\STATE Let count $=0$
\STATE \textbf{for} $\ell=1$ to $(\log{1/\epsilon})/(\log{5/4})$ 
\STATE \quad $M^{\ell}=0$
\STATE \quad \textbf{while} count$ < 2^k$ \textbf{do}
\STATE \quad \quad \textbf{for} $t =$ count$/N_2$ to $(\text{count} + a - 1)/N_2$ \textbf{do in parallel}
\STATE \quad \qquad  $M^{\ell}_{t,N_2} = \parcircuit(G(V, E), k, \mathcal{L}, \mathbf{v}, t, N_2, p, \mathcal{P})$
\STATE \quad \quad \textsc{MpiBarrier}
\STATE \quad \quad $M^{\ell} = M^{\ell} + \sum_{t=\text{count}/N_2}^{(\text{count} + a - 1)/N_2} M_{t,N_2} \mod 2^{k+1}$ using \textsc{MpiReduce}
\STATE \quad \quad count $= \text{count} + a$
\STATE \textbf{if} $M^{\ell}\neq 0$ for some $\ell$
\STATE \quad \textbf{return} \textbf{True}
\STATE \textbf{else} 
\STATE \quad \textbf{return} \textbf{False}
\end{algorithmic}
\end{algorithm}

\begin{algorithm}{}
\small
\caption{\parcircuit{$(G(V, E), k, \mathcal{L}, \mathbf{v}, t, p, \mathcal{P})$}}
\label{alg:parEvaluate} 
\begin{algorithmic}[1]
%\STATE \textbf{Procedure} \parcircuit{$(G(V, E), k, \mathcal{L}, \mathbf{v}, t, p)$}
\STATE \textbf{Input:} Circuit $G(V, E)$, parameter $k$, node levels $\mathcal{L}$, 
random assignment $\mathbf{v}$, iteration number $t$, number of partitions $p$, and
partitioning $\mathcal{P}$
%\STATE Partition $G$ into $p$ parts
%\STATE Let $G^{t,p'}$ be the partition assigned to processor $p'$
%\STATE Let $\mathcal{L}^{t,p'}$ be the corresponding levels assigned to processor $p'$
\STATE
\STATE \textbf{for} processor $p'$ \textbf{do in parallel}
\STATE \quad \textbf{Initialize circuit inputs}
\STATE \quad \textbf{for} node $i \in L^{p'}_{1}$ \textbf{do}
\STATE \qquad $ \val(i) = 1 + (-1)^{v_i^T \cdot t_{\text{bin}}}$

\STATE \quad \textbf{Evaluate the circuit by levels}
\STATE \quad \textbf{for} $s=2$ to $|\mathcal{L}^{p'}|$ \textbf{do}
\STATE \qquad \textbf{for} $i \in L_s^{p'}$ \textbf{do}
\STATE \qquad \quad \textbf{if} $\op(i) = +$ \textbf{then}
\STATE \qquad \qquad $\val(i) = \bar 0$
\STATE \qquad \qquad \textbf{for} $j \in \pred(i)\cap V^{p'}$ \textbf{do}
\STATE \qquad \qquad \quad $\val(i) = \val(i) + \val(j)$
\STATE \qquad \qquad \textbf{for} each incoming message $\langle j, \val(j)\rangle$ \textbf{do}
\STATE \qquad \qquad \quad $\val(i) = \val(i) + \val(j)$
\STATE \qquad \quad \textbf{else} 
\STATE \qquad \qquad $\val(i) = \bar 1$
\STATE \qquad \qquad \textbf{for} $j \in \pred(i)\cap V^{p'}$ \textbf{do}
\STATE \qquad \qquad \quad $\val(i) = \val(i) \cdot \val(j)$
\STATE \qquad \qquad \textbf{for} each incoming message $\langle j, \val(j)\rangle$ \textbf{do}
\STATE \qquad \qquad \quad $\val(i) = \val(i) \cdot \val(j)$
\STATE \qquad \quad  \textbf{Send result to successors in other processors}
\STATE \qquad \quad \textbf{for} $j \in \succ(i) \setminus V^{p'}$ \textbf{do}
\STATE \qquad \qquad \textbf{Send} $\langle i, \val(i)\rangle$
\STATE \textsc{MpiBarrier}
\STATE \textbf{return} $\val($root$(G))$
\end{algorithmic}
\end{algorithm}

Recall the definitions of $\maxload{}_s$ and $\maxdeg{}_s$ corresponding to the partitioning
$\mathcal{P}$ and corresponding level decomposition $\mathcal{L}^j, j=1,\ldots,p$.
Further, let $c_1$ and $c_2$ denote the time for unit computation at any node in $G$
and the unit communication along any edge, respectively, in the Algorithm \parcircuit{}.
The time and communication complexity of algorithm \parmaxwt{} is summarized below
in terms of these parameters.

\begin{theorem}
\label{thm:parmaxwt}
For any $\epsilon\in(0, 1)$,
Algorithm \parmaxwt{} solves the \textsc{$k$-MLD} problem for an
instance $P(x_1,\ldots,x_n)$ with probability at least $1-\epsilon$. The total time for
computation and communication are $O\left(c_1\frac{2^kp}{P}\sum_s \maxload{}_s\log{1/\epsilon}\right)$ 
and $O\left(c_2\frac{2^kp}{P}\sum_s \maxdeg{}_s\log{1/\epsilon}\right)$, respectively.
\end{theorem}
\begin{proof}
First, we argue the correctness.
For each value of $\ell$ in the outer for loop in lines 8-15 of Algorithm \parmaxwt{},
the while loop in lines 10--15 together correspond to the loop in lines 9--10
of the sequential algorithm \maxwt{} (since all the $2^k$ iterations are independent).
Therefore, the quantity $M^{\ell}$ computed in the lines 10--15 corresponds to the
result $M$ in line 12 of \maxwt{}. By Theorem \ref{theorem:kmld2}, if the
\textsc{$k$-MLD} instance $P(x_1,\ldots,x_n)$ has a multilinear term with $k$ variables,
then $\Pr[M^{\ell}\neq 0] = \frac{1}{5}$. This implies that if there is a multilinear term,
$\Pr[M^{\ell} = 0,\ \forall \ell] = (\frac{4}{5})^{(\log{1/\epsilon})/(\log{5/4})}\leq\epsilon$.

Next, we consider the computation and communication time complexity. 
The algorithm \parcircuit{} computes the DAG level by level within each iteration.
Therefore, the computation time for level $s\leq |\mathcal{L}|$ is 
$O(c_1\max_j |L_s^j|) = O(c_1\maxload{}_s)$, which is the maximum time for any processor
for this level. After this level is computed, the results have to be sent on all edges
in the set 
$\{(u, v)\in L_s\times L_{s+1}: u\in L_s^j, v\in L_{s+1}^{j'}\}$, for every pair of
processors $j, j'$. Therefore, the maximum communication time after level $s$ is
$\maxdeg{}_s = \max_{j\neq j'} |\{(u, v)\in L_s\times L_{s+1}: u\in L_s^j, v\in L_{s+1}^{j'}\}|$.
This implies the bounds on the total computation and communication steps.
\end{proof}

\subsection{Partition and Load Balancing}
\label{sec:partition}

From Theorem \ref{thm:parmaxwt}, the performance of Algorithm \parmaxwt{} depends crucially
on the partitioning. Specifically, the partition should be ``vertical'', i.e., one which achieves
load balance across each level and also minimizes the edge cut, which is formalized by
\[
\text{cost}(\mathcal{P}) = \sum_s (c_1\maxload{}_s + c_2\maxdeg{}_s)
\]

However, finding a partitioning $\mathcal{P}$ that minimizes the above objective
$\text{cost}(\mathcal{P})$ is NP-hard, in general, as summarized below. The proof
is omitted for brevity.

\begin{lemma}
\label{lemma:partition}
Given a DAG $G=(V, E)$ and per-unit computation and communication costs $c_1$ and $c_2$,
respectively, finding a partitioning $\mathcal{P}$ with the minimum cost is NP-hard, in general.
\end{lemma}

In light of Lemma \ref{lemma:partition}, we consider two heuristics: finding balanced
partitions in each level, and combining them, as well as METIS \cite{karypis:sijsc99}.

%We partition the circuit ``vertically"; that is, each processor gets a subset of nodes for each level in $\mathcal{L}$. Let $f(i)$ be the cost of evaluating node $v$. Then, we want to find a partition, such that
%$$
%\sum_{v \in V^p} f(i) \approx \frac{1}{p}\sum_{i \in V} f(i).
%$$

%\subsection{Analysis of Algorithm \parmaxwt{}}
%\subsubsection{Time Complexity}
%The \textbf{while} loop in lines 8--13 of \parmaxwt{} runs for at most $\frac{2^k}{P / p}$ iterations. For each iteration, each partition does work in the order $O(\frac{1}{p}\sum_{i \in V} f(i))$, so the total running time is $O(\frac{1}{p}(2^k f(V)))$.
%\subsubsection{Number of Messages}
%We assume that we have $P$ processors available and that $G$ does not fit in the memory of a single computing node. We have to use $p \leq P$ processors to store the graph in memory. We propose an MPI algorithm with roughly the following steps:

%\begin{enumerate}
%\item \parmaxwt{} Algorithm \ref{alg:parallel-kMLD}
%\item Determine $p$, the number of processors needed to store the graph.
%\item We have to run the loop from line 9 of \maxwt{}. Each of the $2^k$ iteration is independent of the others, so we can run $\lfloor{P/p}\rfloor$ iterations in parallel.
%\item \parcircuit{} Algorithm \ref{alg:parEvaluate}
%\item For each iteration $t$, we first partition the circuit into $p$ parts, and we evaluate these parts in parallel.
%\item For each partition $p'$, we evaluate the circuit by levels, as in \maxwt{}. 
%\item For each node $i$ that we evaluate, we compute $m(i)$ using the $m(j)$ value for each predecessor $j$. if a predecessor $j$ is in the same partition $p'$, then, we can simply read $m(j)$, as in the sequential algorithm. For every predecessor $j$ in a different partition, $j$ has to send a message with $m(j)$, introducing a communication overhead.
%\item We aggregate the value at root$(G)$ over all $2^{k}$ iterations to get the final solution.
%\end{enumerate}

%\noindent
%\textbf{Algorithm \ref{alg:parallel-kMLD}}: \parmaxwt{}\\
%1. Determine $p$, the number of processors needed to store the graph.\\
%2. We have to run the loop from line 9 of \maxwt{}. Each of the $2^k$ iteration is independent of the others, so we can run $\lfloor{P/p}\rfloor$ iterations in parallel.\\
%\textbf{Algorithm \ref{alg:parEvaluate}}: \parcircuit{} \\
%3. For each iteration $t$, we first partition the circuit into $p$ parts, and we evaluate these parts in parallel.\\
%4. For each partition $p'$, we evaluate the circuit by levels, as in \maxwt{}. \\
%5. For each node $i$ that we evaluate, we compute $m(i)$ using the $m(j)$ value for each predecessor $j$. If a predecessor $j$ is in the same partition $p'$, then, we can simply read $m(j)$, as in the sequential algorithm. For every predecessor $j$ in a different partition, $j$ has to send a message with $m(j)$, introducing a communication overhead.\\
%6. We aggregate the value at root$(G)$ over all $2^{k}$ iterations to get the final solution.

\section{Applications of parallel multilinear detection}
\label{sec:applications}

Multilinear detection has a broad class of applications, two of which we discuss here.
The first involves finding paths and trees in a graph---Koutis et al.
\cite{DBLP:journals/talg/KoutisW16} showed that these problems can be solved using
multilinear detection. We describe how Algorithm \parmaxwt{} can be adapted
to solve this problem. The second problem involves anomaly detection using the
approach known as graph scan statistics. This was solved sequentially using 
the color coding technique in \cite{cadena:sdm17}; we show how this entire class
of problems can be solved by adapting Algorithm \parmaxwt{}.
As before, $G = (V, E)$, denotes a graph where $V$ is a set of $n$ vertices or nodes, and 
$E$ is a set of $m$ edges. Let $\nbr{v}=\{u: (u,v)\in E\}$ denote the set of neighbors
of node $v$.

\subsection{Finding Paths and Trees}
\label{sec:apps-trees}

%We consider a more general, weighted version of the problem. We assume a weight $w(v)$
%for each node $v\in V$.
Given a graph $G=(V, E)$ with $n=|V|$, $m=|E|$, and a subgraph $H=(V_H, E_H)$, with $k=|V_H|$,
the basic subgraph isomorphism problem involves finding a mapping $f:V_H\rightarrow V$
such that $(i, j)\in E_H$ if and only if $(f(i), f(j))\in E$. 

\begin{problem} ($k$-Tree)
\label{prob:trees}
Given a weighted graph $G=(V, E)$ with a weight vector $\mathbf{w}$, and a tree
denoted by $H=(V^H, E^H)$ with $|V^H|=k$, the objective is to determine if there exists
an embedding of $H$ in $G$.
\end{problem}

We show below that problem \ref{prob:trees} can be solved by Algorithm \parmaxwt{} 
by formalizing this as a multilinear detection problem. We first consider the case
where $H$ is a path of length $k$, for notational simplicity.
Let $x_v$ denote a variable associated with each node $v\in V$.
We define poynomials $P_v(i)$ for all $v\in V$, $i\leq k$ in the following manner.

\begin{itemize}
\item
$P_v(1) = x_v$ for all $v\in V$
\item
For $i>1$,
$P_v(i) = \sum_{i'<i} \sum_{u\in\nbr(v)} P_u(i')P_v(i-i')$
\item
Define the polynomial $P(x_1,\ldots,x_n) = \sum_v P_v(k)$
\end{itemize}

It can be verified that the graph $G$ has a path of length $k$ if and only if the
polynomial $P(x_1,\ldots,x_n)$ has a multilinear term.


\begin{figure}[h]
\includegraphics[width=0.4\textwidth]{img/trees.pdf}
\caption{
\small
Tree $H$ with $\myroot(H)=1$. It is decomposed into trees $H_1$ and $H_2$ by
removing the edge $(1, 2)$. $\myroot(H_1)=1$ and $\myroot(H_2)=2$.
%\vspace{-0.2in}
}
\label{fig:trees}
\end{figure}
Next, we consider the case where $H$ is a tree. We consider the tree to be rooted,
and let $\myroot(H)$ be the root node, selected arbitrarily. We consider a hierarchical
structure among subtrees of $H$ in the following manner: consider any node $u\in\nbr(\myroot(H))$.
Let $H_1$ and $H_2$ denote the subtrees obtained upon deleting the edge $(u, \myroot(H))$,
with $\myroot(H)\in H_1$ and $u\in H_2$. We set $\myroot(H_1)=\myroot(H)$ and $\myroot(H_2)=u$.
This process is illustrated i Figure \ref{fig:trees}.
The subtrees $H_1$ and $H_2$ are further partitioned in a recursive manner, till
all trees have a single node. For an intermediate tree $H'$ in this process,
let $\textsc{child}_1(H')$ and $\textsc{child}_2(H')$ denote the two child trees
resulting from the tree $H'$. Let $\textsc{parent}(H')=H''$ be the tree such that
either $\textsc{child}_1(H'')= H'$ or $\textsc{child}_2(H'') = H'$.
We define the polynomials $P_v(H')$, which will correspond to all layouts (not necessarily
isomorphisms) of $H'$ with $\myroot(H')=v$, in the following manner:
\begin{itemize}
\item
If $H'$ consists of a single node, $P_v(H') = x_v$
\item
Else, 
$P_v(H') = \sum_{u\in\nbr(v)} P_v(H'_1)P_u(H'_2)$, where
$H'_1$ and $H'_2$ denote $\textsc{child}_1(H')$ and $\textsc{child}_2(H')$, respectively.
\item
Finally, we have
$P(x_1,\ldots, x_n)= \sum_v P_v(H)$
\end{itemize}

More generally, we consider a weighted version of the problem, where the goal is to
find an embedding of the maximum weight. 

\subsection{Anomaly Detection Using Graph Scan Statistics}
\label{sec:apps-scanstat}

We use the notation of \cite{cadena:sdm17} here. We assume each node $v\in V$ has two associated values,
which vary with time (we will not show the time, to avoid complicating the notation):
(1) a \emph{baseline count}, $b(v)$, which indicates the count that we 
expect to see at the node $v$---e.g., the number of people in a county corresponding to node $v$---and
(2) an \emph{event count} or \emph{weight}, $w(v)$, which indicates how many occurrences of an event 
of interest are seen at the node---e.g., the number of cases of a disease in a county.

Graph scan statistics are among the most commonly used methods for detecting anomalies or ``hotspots" in 
networked data \cite{Speakman-14,leiserson2015pan, hansen2016finding, neil2013scan, chen2014non}. 
Informally, this approach formalizes anomaly detection as a hypothesis testing problem.
Under the null hypothesis $H_0$, it is \emph{business as usual}, and the event counts for all nodes are generated proportionally to their baseline counts. Under the alternative hypothesis $H_1(S)$, counts of a majority of
the vertices are generated (again) with rate proportional to the baseline counts, but there exists a small connected subset
$S \subseteq V$ of vertices for which the counts are generated at a higher rate than expected.
Then, the goal is to find a set of vertices $S$ that maximizes an appropriate scan statistic function $F(S)$, typically a log-likelihood ratio that compares event counts to baseline counts. We define a scan statistic in terms of the event and baseline counts of a node set:
$$
F(S) = F(W(S), B(S), \mathbf{\theta}),
$$
where $W(S) = \sum_{v \in S} w(v)$ is the total event count or \emph{weight} of $S$, $B(S) = \sum_{v \in S} b(v)$ is the baseline count of the set, and $\theta$ represents possible additional arguments to $F$.

Depending on the assumptions that are satisfied by the data, there are two broad types of scan statistics: parametric and non-parametric. 
An example of a non-parametric function is the Berk-Jones scan statistic (BJ) \cite{Berk-79} used for civil unrest events and network intrusion detection~\cite{chen2014non,mcfowland2013fast}. In this setting, each node $v$ has a $p$-value $p(v) \in [0,1]$, and, for a significance level $\alpha$, the event count $w(v)$ is 1 if $p(v) < \alpha$ (i.e., the node is significant) and 0 otherwise, and the baseline count is $b(v) = 1$ for all nodes. This scan statistic is defined as
{\scriptsize
$$
\max_{\alpha \leq \alpha_{max}}B(S) \left[\frac{W(S)}{B(S)} \log\left(\frac{\frac{W(S)}{B(S)}}{\alpha}\right) + \left(1 - \frac{W(S)}{B(S)}\right) \log\left(\frac{1 - \frac{W(S)}{B(S)}}{1 -\alpha}\right) \right],
$$}
with $\mathbf{\theta} = \alpha_{max}$.
A well-known example of a parametric function is the Kulldorff scan statistic commonly used in disease surveillance \cite{kulldorff_spatial_1997,Duczmal06,kulldorff2003power,neill-jss12}, which is defined as 
{\scriptsize $$
W(S) \log\left(\frac{W(S)}{B(S)}\right) + (W(V) - W(S)) \log\left(\frac{W(V) - W(S)}{B(V) - B(S)}\right) - C(V) \log\left(\frac{W(V)}{B(V)}\right),
$$}
with $\mathbf{\theta} = (W(V),B(V))$. 
A number of other scan statistics are discussed in \cite{cadena:sdm17}. Our methods will extend to
all of those. We also note that the suitability of scan statistics depends on the application and the assumptions
underlying the dataset. We refer to
\cite{margai2003community, neill-jss12, kulldorff_spatial_1997, neill2007nonparametric}
for a more detailed discussion of the advantages and limitations
of these approaches, since this is not the focus of our paper.

\noindent
\textbf{Problem Formulation} 
The graph anomaly detection problem can be posed as the following constrained optimization problem,
which has been shown to be NP-hard, in general \cite{cadena:sdm17}.

\begin{problem}
\label{prob:macs}
Given a graph $G=(V, E)$, a scan statistic $F(\cdot)$, the associated counts for vertices---$\mathbf{w}$ and $\mathbf{b}$---and a parameter $k$, find a connected subset $S\subseteq V$ that maximizes $F(S) = F(W(S), B(S), \theta)$ with $B(S) \leq k$.
\end{problem}

\noindent
\textbf{Reduction to multilinear detection on a DAG.}
We define the sets $K=\{1,2 \ldots, k\}$ where $k$ is a \emph{size parameter}, 
and $R=\{0,1,2,\ldots, r\}$, where $r$ is a \emph{weight parameter}. Let $w(v)$ denote the
weight for each node $v\in V$; the weights are binned by considering powers of $(1+\epsilon)$, for an error parameter $\epsilon>0$: if $w(v)\in [(1+ \epsilon)^{j-1}, (1+ \epsilon)^{j})$, we will sometimes say that $v$ is
in the weight group $j$; this definition is extended to the weight of a subgraph.

For each node $v$, we define a variable $x_v$, and we construct a polynomial over the set of variables $\{x_v: v \in V\}$. Every term---i.e., monomial---in this polynomial will represent a connected subgraph of size at most $k$ and weight at most $(1+ \epsilon)^r$. 
For $i \in K$ and $j \in R$, let $P_v(i,j)$ be the polynomial corresponding to a subgraph (1) containing node $v$, (2) of size $i$, and (3) total weight in 
$[(1+ \epsilon)^{j-1}, (1+ \epsilon)^{j})$. $P_v(i,0)$ represents subgraphs of weight 0.
The following recurrence relations describe how the polynomials $P_v(i, j)$ are computed:
\begin{itemize}
\item
$P_v(i, j) = \bar{0}$ for $i \in K$, $j \in R$
\item
$P_v(1, j) = x_v$ for all $v \in V$, $j = \lceil \log_{(1 + \epsilon)} (w(v) + 1)\rceil$
\item
for $v \in V$, $i = 2$ to $k$, $j = 0$ to $r$:
\begin{itemize}
\item
if $j=0$: $P_v(i,0) = \sum_{u \in \nbr(v)} \sum_{i' = 1}^i (P_v(i',0) \cdot P_u(i-i', 0))$ 
\item
if $j=1$:
$P_v(i,1) = \sum_{u \in \nbr(v)} \sum_{i' = 1}^i (P_v(i', 1) \cdot P_u(i-i', 0) + (P_v(i', 0) \cdot P_u(i-i', 1))$
\item
if $j\geq 2$:
$P_v(i,j) = \sum_{u \in \nbr(v)} \sum_{i' = 1}^i (P_v(i', j - 1) \cdot P_u(i-i', j - 1) +
P_v(i', j) \cdot P_u(i-i', 0) + P_v(i', 0) \cdot P_u(i-i', j))$
\end{itemize}
\item
$P(i,j) = \sum_v P_v(i,j)$ for $i \in K$, $j \in R$
\end{itemize}

The above computation can be represented as a polynomial described by a DAG, in which the
nodes corresponding to $P_v(i, j)$ are the intermediate nodes, which are computed by ``$\cdot$'' and ``$+$''
operations.

% !TEX root = ./multilinear.tex
\section{Experiments}
\label{sec:experiments}
We evaluated the performance of the proposed parallel algorithms for parallel speedup and scalability. In particular, our experiments address the following questions:

%In this regard we varied the graph size, motif size (parameter $k$ in Algorithm~\ref{alg:parallel-kMLD} and Algorithm\textcolor{red}{TODO - ref to parallel multi linear scan}), and parallelism. 

\begin{enumerate}
\item 
How does performance scale with the number of processors for different graphs for the specific problems of finding paths and graph scan statistics? (Section \ref{sec:scaling})
\item
How does performance varies with parallelism over iterations and graph nodes? (Section~\ref{sec:n1n}
\item
How does performance of finding paths compare with the popular FASCIA~\cite{6877274} subgraph counting implementation? (Section~\ref{sec:vsfascia})
\item
How does performance compare with the popular distributed vertex programming model,
Giraph? (Section \ref{sec:compare-giraph})
%\item
%How do our methods compare with those based on color coding?
\item
What are settings where the scaling to larger subgraph size gives new results? (Section \ref{sec:traffic})
\end{enumerate}

\subsection{Experimental Setup}
\subsubsection{Hardware}
Experiments were conducted on Juliet, an Intel Haswell HPC cluster. Up to 32 nodes were used for the evaluation, where each node has 36 cores (2 sockets x 18 cores each). A node consists of 128GB of main memory and 56Gbps Infiniband interconnect. We also tested on another HPC cluster, Shadowfax-Haswell, where we used 32 nodes each with 32 cores (2 sockets x 16 cores each). Memory and interconnect of this cluster are similar to those of Juliet.

\subsubsection{Datasets}
For the two applications (paths and scan statistics), we evaluate our algorithms in datasets from different domains, such as social networks, citation networks, and road networks. A summary of the datasets is provided in Table \ref{table:datasets}. All the networks may be found in the SNAP repository \cite{snapnets}. In addition, we perform experiments in two Erdos-Renyi networks of 1 and 10 million nodes with an expected number of edges of $n\log n$, where $n$ is the number of nodes.

\begin{table}[ht]
\centering \caption{\small Datasets used in our experiments}
\vspace{-.1in}
\label{table:datasets}
%\resizebox{\columnwidth}{!}{
%\begin{scriptsize}
\begin{tabular}{|l|r|r|}
\hline
\textbf{Dataset}  & \textbf{Nodes ($\times 10^6$)} & \textbf{Edges ($\times 10^6$)} \\
\hline
miami & 2.1 & 51.5\\
\hline
com-Orkut  & 3.1 & 234.3\\
\hline
random-1e6 & 1 & 13.8\\
\hline
random-1e7 & 10 & 161.8\\
\hline
\end{tabular}
%\end{scriptsize}
%}
\end{table}

\begin{figure*}[!htb]
    \centering
    \begin{minipage}{0.23\textwidth}
        \centering        
        \includegraphics[width=1\columnwidth]{img/kpath-N1N/1mil-k6-BS1.png}
        \caption{\textcolor{blue}{kpath-1mil-k6-BS title}}
        \label{fig:fig-kpath-1mil-k6-BS1.png}
    \end{minipage}
    \hspace{0mm}
    \begin{minipage}{0.23\textwidth}
        \centering
        \includegraphics[width=1\columnwidth]{img/kpath-N1N/1mil-k6-BSMax.png}
        \caption{\textcolor{blue}{kpath-1mil-k6-BSMax}}
        \label{fig:fig-kpath-1mil-k6-BSMax.png}
    \end{minipage}  
    \hspace{0mm}
    \begin{minipage}{0.23\textwidth}
        \centering
        \includegraphics[width=1\columnwidth]{img/kpath-N1N/10mil-k6-BS1.png}
        \caption{\textcolor{blue}{kpath-10mil-k6-BS1}}
        \label{fig:fig-kpath-10mil-k6-BS1.png}
    \end{minipage}   
    \hspace{0mm}
    \begin{minipage}{0.23\textwidth}
        \centering
        \includegraphics[width=1\columnwidth]{img/kpath-N1N/10mil-k6-BSMax.png}
        \caption{\textcolor{blue}{kpath-10mil-k6-BSMax}}
        \label{fig:fig-kpath-10mil-k6-BSMax.png}
    \end{minipage}   
    \hspace{0mm}
    \begin{minipage}{0.23\textwidth}
        \centering
        \includegraphics[width=1\columnwidth]{img/kpath-N1N/orkut-k6-BS1.png}
        \caption{\textcolor{blue}{kpath-orkut-k6-BS1}}
        \label{fig:fig-kpath-orkut-k6-BS1.png}
    \end{minipage}   
    \hspace{0mm}
    \begin{minipage}{0.23\textwidth}
        \centering
        \includegraphics[width=1\columnwidth]{img/kpath-N1N/orkut-k6-BSMax.png}
        \caption{\textcolor{blue}{kpath-orkut-k6-BSMax}}
        \label{fig:fig-kpath-orkut-k6-BSMax.png}
    \end{minipage}   
    \hspace{0mm}
    \begin{minipage}{0.23\textwidth}
        \centering
        \includegraphics[width=1\columnwidth]{img/kpath-N1N/miami-k6-BS1.png}
        \caption{\textcolor{blue}{kpath-miami-k6-BS1}}
        \label{fig:fig-kpath-miami-k6-BS1.png}
    \end{minipage}   
    \hspace{0mm}
    \begin{minipage}{0.23\textwidth}
        \centering
        \includegraphics[width=1\columnwidth]{img/kpath-N1N/miami-k6-BSMax.png}
        \caption{\textcolor{blue}{kpath-miami-k6-BSMax}}
        \label{fig:fig-kpath-miami-k6-BSMax.png}
    \end{minipage}   
\end{figure*}

% [Saliya] I think we don't need this section on
% independence of iterations because now we don't
% run these as separate instances. Everything is 
% automatically done within the program.
% \subsection{Independence of Iterations}
% \label{sec:independence}
% First, we examine the extent to which the $2^k$ iterations of the outer loop in
% Algorithm \parmaxwt{} are independent and take the same time. Figure \ref{fig:indep} shows the running time as the number of processors $N$ is increased for the random network of 1 million nodes. This figure implies that the running time is almost invariant with the
% number of parallel iterations that are run simultaneously. We observed similar results with com-Orkut graph as shown in Figure~\ref{fig:indep-orkut}. We use this insight to estimate the total running time for other settings. 

% \begin{figure}[!htpb]
% \includegraphics[width=0.4\textwidth]{img/fig-random-1mil-timeperiteration.pdf}
% \caption{Time per iteration with varying parallel units for random 1 million nodes network.}
% \label{fig:indep}
% \end{figure}

% \begin{figure}[!htpb]
% \includegraphics[width=0.4\textwidth]{img/fig-orkut-timeperiteration.pdf}
% \caption{Time per iteration with varying parallel units for Orkut network.}
% \label{fig:indep-orkut}
% \end{figure}

\subsection{Scalability with $N$ and $N_1$}
\label{sec:n1n}
The multilinear detection based parallel $k$-Path and Multilinear Scan algorithms exhibit two levels of parallelism: vertex and iterations. On one hand, the $2^k$ iterations are pleasingly parallel except a reduction at the end. The parallel vertex computation within each iteration, on the other hand, requires message passing between neighbors for $k-1$ steps. 

\textcolor{blue}{TODO - change bs to N2}

Given these two levels of parallelism and a total of $N$ processes, we can split the $2^k$ iterations among $a=\lfloor N/N_1 \rfloor$ parallel instances. Each instance decomposes the graph across $N_1$ processes and performs the $k$-Path computation in parallel for $\lfloor 2^k/a \rfloor$. To reduce the communication over computation cost, the algorithm packs a user defined $bs$ number of iterations into one computation step, so each parallel instance only has to perform $\lfloor 2^k/(a*bs) \rfloor$ compute and communication phases. 

To illustrate this with an example, consider the case of $k=6$, $N=128$, $N_1=32$, and $bs=8$. The total number of iterations is $2^k = 64$. The number of parallel instances is $128/32 = 4$. Each instances only needs to run $64/4 = 16$ iterations. Since $bs=8$, the $16$ iterations can be completed in just $16/8 = 2$ phases.

A note on $bs$: increasing this value, for example $bs=16$ in the previous case, could have finished the entire program in one compute and communicate phase. This yields to higher parallel efficiency as the overhead of communication to computation is reduced. However, it increases the message size by $bs$ factor. Depending on the number of total processes, MPI may fail to accommodate large message sizes requiring to reduce  $bs$ such that \textcolor{blue}{TODO}. 

With these two levels of parallelism we expect to see a minimal

\textcolor{blue}{Saliya TODO - fix below}
We evaluate the total time to complete for $k$-Path and Multilinear 
Scan algorithms as a function of $N_1$, the number of processes used to run 
each of the $2^k$ circuit evaluations. With a total of $N$ processes, we can run 
$\lfloor N / N_1 \rfloor$ iterations in parallel. Figure \ref{fig:time-with-p} shows the total time for both applications varying $N$ and $N_1$. First, we note that, as $N$ becomes larger, the total running time decreases for all values of $N_1$, which is expected. Second, we obtain the best performance by assigning around 8 parallel units to process an iteration, regardless of the total parallel units.

%\subsection{Comparison with Giraph}
%We now compare our \parmaxwt{} with an implementation of Algorithm \ref{alg:multilinear-detect} in Giraph, which is a ``think like a vertex" computation framework. In Figure \ref{fig:giraph-comparison}, we show the running time of \parmaxwt{} and the Giraph implementation as a function of $k$. Giraph uses all the available parallel units for each of the $2^k$ circuit evaluation. Therefore, for a fair comparison, we report the time per iteration of both implementations---even though \parmaxwt{} runs multiple evaluations in parallel. Our MPI algorithm shows much better scalability with $k$, giving as much as 5x speed up over Giraph in the LiveJournal network.
%


% \begin{figure*}[ht]
%   \centering
%   \begin{subfigure}[b]{\textwidth}
%     \centering
%     \includegraphics[width=0.49\textwidth]{img/total-time-with-p-live-journal.pdf}
%     \includegraphics[width=0.49\textwidth]{img/total-time-with-p-random-er-1e6.pdf}
%     \caption{\label{fig:np-kpath} k-Path}
%   \end{subfigure}\\
  
%   \begin{subfigure}[b]{\textwidth}
%     \centering
%     \includegraphics[width=0.49\textwidth]{img/total-time-with-p-live-journal-multilinear.pdf}
%     \includegraphics[width=0.49\textwidth]{img/total-time-with-p-random-er-1e6-multilinear.pdf}
%     \caption{\label{fig:np-multilinear} Multilinear Scan}
%   \end{subfigure}%
%   \caption{Scalability of our MPI algorithms for $k$-Path (\ref{fig:np-kpath}) and Multilinear Scan (\ref{fig:np-multilinear}) as a function of the total paralellism ($N$) and the units assigned to a single circuit evaluation ($p$).}
%  \label{fig:time-with-p}
% \end{figure*}

\subsection{Strong Scaling and Speedup}
\label{sec:scaling}
In Figure \ref{fig:scaling}, we show the speed up and strong scaling achieved by increasing the total parallelism. We obtain higher speed up for $k$-Path, possibly because the communication overhead for this program is lower than for Multilinear Scan. In both cases, we the scaling factor is better for the LiveJournal network.

\begin{figure*}[ht]
  \centering
  \begin{subfigure}[b]{\textwidth}
    \centering
    \includegraphics[width=0.45\textwidth]{img/speedup-kpath.pdf}
    \includegraphics[width=0.45\textwidth]{img/strong-scaling-kpath.pdf}
    \caption{\label{fig:scale-kpath} k-Path}
  \end{subfigure}\\
  
  \begin{subfigure}[b]{\textwidth}
    \centering
    \includegraphics[width=0.45\textwidth]{img/speedup-multilinear.pdf}
    \includegraphics[width=0.45\textwidth]{img/strong-scaling-multilinear.pdf}
    \caption{\label{fig:scale-multilinear} Multilinear Scan}
  \end{subfigure}%
  \caption{Speedup and strong scaling of our MPI algorithms for $k$-Path (\ref{fig:scale-kpath}) and Multilinear Scan (\ref{fig:scale-multilinear}).}
 \label{fig:scaling}
\end{figure*}

\subsection{Comparision with Giraph}
\label{sec:compare-giraph}
%Performance for varying k=1 to k=16 in increments of 2
%This can include both random-er and snap graphs
We now compare our \parmaxwt{} with an implementation of Algorithm \ref{alg:multilinear-detect} in Giraph, which is a ``think like a vertex" computation framework. In Figure \ref{fig:giraph-comparison}, we show the running time of \parmaxwt{} and the Giraph implementation as a function of $k$. Giraph uses all the available parallel units for each of the $2^k$ circuit evaluation. Therefore, for a fair comparison, we report the time per iteration of both implementations---even though \parmaxwt{} runs multiple evaluations in parallel. Our MPI algorithm shows much better scalability with $k$, giving as much as 5x speed up  over Giraph in the LiveJournal network. We also implemented a version using Spark's GraphX library, but the communication cost associated with the dataflow model in Spark was prohibitively expensive for this algorithm. Giraph, on the other hand, follows a peer-to-peer communication model, similar to MPI, making it a better choice than GraphX.

\begin{figure}[!htpb]
\includegraphics[width=0.4\textwidth]{img/giraph-comparison.pdf}
\caption{Running time of \parmaxwt{} as a function of $k$ compared to a Giraph implementation.}
\label{fig:giraph-comparison}
\end{figure}

\subsection{Congested Highways Clusters in Road Networks}
\label{sec:traffic}
We apply our algorithm for scan statistics to find clusters with unexpectedly low-moving traffic in the highway network of Los Angeles County\footnote{\url{http://pems.dot.ca.gov/}}. Nodes in the graph are sensors next to the road that record the average speed and the number of vehicles passing through. We have 30-minute snapshots for May 2014. We assume that the average speed recorded by each sensor follows a normal distribution. Then, the $p$-value of a node $v$ is the cumulative distribution function of a normal distribution with mean  $x_v^{[1,t-1]}$ and standard deviation $\sigma_v^{[1,t-1]}$, where $x_v^{[1,t-1]}$ and $\sigma_v^{[1,t-1]}$ are, respectively, the sample mean and standard deviation for node $v$ from snapshots $1$ to $t-1$.

We use our algorithm with $k=12$ on this dataset. In Figure \ref{fig:traffic}, we show with blue dots highway segments that our algorithm identifies as having \emph{unexpectedly} low average speed during rush hour (16:00 to 19:00) on Friday May 9, 2014. These segments are not necessarily the ones with most congestion. For instance, the center of Los Angeles city has higher congestion; however, such activity is normal on Friday afternoons according to the previous snapshots. The clusters shown in the map are selected because they have significantly lower average speeds than in previous observations.  

\begin{figure}[!htpb]
%\frame{\includegraphics[width=0.48\textwidth,trim={0 4cm 0 4cm},clip]{img/traffic-network.pdf}}
\includegraphics[width=0.48\textwidth]{img/traffic-example.png}
\caption{Discovering highway segments with unexpected congesion in the Los Angeles road network.}
\label{fig:traffic}
\end{figure}

% !TEX root = ./multilinear.tex
\section{Related Work}
\label{sec:related}

There is a huge literature on a variety of subgraph analysis problems, arising
out of a number of applications, such as bioinformatics, security,
social network analysis, epidemiology and finance (see  \cite{akoglu2014graph} for a survey).
We discuss some of three main directions here: 
subgraph isomorphism and clique
enumeration (for which parallel algorithms exist), and anomaly detection (for which
there has been limited work on parallel algorithms).

Given a graph $G=(V, E)$ with $n=|V|$, $m=|E|$, and a subgraph $H=(V_H, E_H)$, with $k=|V_H|$,
the basic subgraph isomorphism problem involves finding a mapping $f:V_H\rightarrow V$
such that $(i, j)\in E_H$ if and only if $(f(i), f(j))\in E$. This is
a very computationally challenging problem. The frequent subgraph detection
problem involves finding subgraph isomorphisms (also referred to as embeddings) 
having frequency higher than a threshold; other variations involve finding
non-overlapping embeddings.
Parallel approaches for this problem involve a ``bottom-up'' candidate generation approach,
combined with careful pruning, which builds embeddings of larger subgraphs using all possible
embeddings of smaller subgraphs\cite{hamid2016scalemine, elsidy:vldb14}. While these results
allow scaling to very large networks with millions of nodes, they give no guarantees on
the performance. Our work is more closely related to the use of the color coding technique
for finding tree-like subgraphs \cite{alon2008biomolecular, huffner2008algorithm}, which
guarantees a fully polynomial time approximation to the number of embeddings with
running time and space of $O(2^ke^km\log{n})$ and $O(2^km)$, respectively.
This has been parallelized using MapReduce \cite{zhao2012sahad} and 
OpenMP \cite{slota:icpp13, slota:ipdps14}, enabling subgraph counting in graphs with
tens of millions of nodes with rigorous guarantees. Slota et al. \cite{slota:icpp13, slota:ipdps14}
use threading and techniques for reducing the memory footprint of the color coding
dynamic programming tables, in order to scale. Our approach uses algebraic methods
for which the memory scales as $O(k)$ instead of $O(2^k)$, and the worst case
running time scales as $O(2^k)$ instead of $O(2^ke^k)$, which gives the improved performance.

Another area where parallel algorithms have been developed is for dense subgraph enumeration.
This is a very challenging problem, since finding the largest clique is NP-hard to
approximate even within an $O(n^{1-\epsilon})$ factor for any $\epsilon>0$.
There are several implementations for finding maximal cliques in parallel by
careful partitioning, pruning and backtracking heuristics
\cite{schmidt2009scalable, zhao2016parallel, aparicio:ispa14, cheng:kdd12, du:mcd09}.
Our results do not extend to the clique enumeration problem.

Finally, anomaly detection is a broad topic, and there has been some work
on parallel algorithms, e.g., \cite{shanbhag:icccn08}. Our work is most related to
the approach known as graph scan statistics, which involves finding connected subgraphs
that optimize specific functions that model underlying processes about the data.
While there exist a number of heuristics, the work of \cite{cadena:sdm17} gives the
first rigorous methods for optimizing most scan statistics using the color coding technique.
However, all these methods are sequential.

% !TEX root = ./multilinear.tex
\section{Conclusions}
\label{sec:conc}
We present a new class of distributed MPI-based  algorithms for various subgraph detection problems based on the recent multilinear detection technique.
Even with a naive partitioning scheme, we observe significant performance improvement over the state-of-the-art color coding based methods.
Our algorithms are conceptually much simpler than color coding and scale to subgraphs of size 18, which hasn't been done before. Our method combines parallelization of two different parts of the sequential multilinear algorithm, with a batched communication, and a data structure that gives cache locality.

\noindent
\textbf{Acknowledgements.} This work was partially supported by the following grants: DTRA CNIMS Contracts HDTRA1-11-D-0016-0010, HDTRA1-17- 0118, and NSF grants IIS-1633028, ACI-1443054.

% State of the art parallel algorithms for various subgraph detection problems are based on the color coding technique, which yields algorithms with running time and space complexity proportional exponential on a solution size $k$. Here, we have presented algorithms based on a more recent technique from the parameterized complexity literature, multilinear detection. This methodology gives us improved bounds on memory and time over color coding. We propose an MPI algorithm for multilinear detection for general polynomials, and we show applications to two important problems, $k$-path and anomaly detection via scan statistics. We also show that finding a partitioning with minimum cost for the problem discussed here is NP-Hard, and we leave the development of partitioning heuristics as a topic of future work.

\newpage
\bibliographystyle{plain}
\bibliography{refs}
\bibliography{refs-exp}

\end{document}
