% !TEX root = ./multilinear.tex
\section{Related Work}
\label{sec:related}

There is a huge literature on a variety of subgraph analysis problems, arising
out of a number of applications, such as bioinformatics, security,
social network analysis, epidemiology and finance (see  \cite{akoglu2014graph} for a survey).
We discuss some of three main directions here: 
subgraph isomorphism and clique
enumeration (for which parallel algorithms exist), and anomaly detection (for which
there has been limited work on parallel algorithms).

Given a graph $G=(V, E)$ with $n=|V|$, $m=|E|$, and a subgraph $H=(V_H, E_H)$, with $k=|V_H|$,
the basic subgraph isomorphism problem involves finding a mapping $f:V_H\rightarrow V$
such that $(i, j)\in E_H$ if and only if $(f(i), f(j))\in E$. This is
a very computationally challenging problem. The frequent subgraph detection
problem involves finding subgraph isomorphisms (also referred to as embeddings) 
having frequency higher than a threshold; other variations involve finding
non-overlapping embeddings.
Parallel approaches for this problem involve a ``bottom-up'' candidate generation approach,
combined with careful pruning, which builds embeddings of larger subgraphs using all possible
embeddings of smaller subgraphs\cite{hamid2016scalemine, elsidy:vldb14}. While these results
allow scaling to very large networks with millions of nodes, they give no guarantees on
the performance. Our work is more closely related to the use of the color coding technique
for finding tree-like subgraphs \cite{alon2008biomolecular, huffner2008algorithm}, which
guarantees a fully polynomial time approximation to the number of embeddings with
running time and space of $O(2^ke^km\log{n})$ and $O(2^km)$, respectively.
This has been parallelized using MapReduce \cite{zhao2012sahad} and 
OpenMP \cite{slota:icpp13, slota:ipdps14}, enabling subgraph counting in graphs with
tens of millions of nodes with rigorous guarantees. Slota et al. \cite{slota:icpp13, slota:ipdps14}
use threading and techniques for reducing the memory footprint of the color coding
dynamic programming tables, in order to scale. Our approach uses algebraic methods
for which the memory scales as $O(k)$ instead of $O(2^k)$, and the worst case
running time scales as $O(2^k)$ instead of $O(2^ke^k)$, which gives the improved performance.

Another area where parallel algorithms have been developed is for dense subgraph enumeration.
This is a very challenging problem, since finding the largest clique is NP-hard to
approximate even within an $O(n^{1-\epsilon})$ factor for any $\epsilon>0$.
There are several implementations for finding maximal cliques in parallel by
careful partitioning, pruning and backtracking heuristics
\cite{schmidt2009scalable, zhao2016parallel, aparicio:ispa14, cheng:kdd12, du:mcd09}.
Our results do not extend to the clique enumeration problem.

Finally, anomaly detection is a broad topic, and there has been some work
on parallel algorithms, e.g., \cite{shanbhag:icccn08}. Our work is most related to
the approach known as graph scan statistics, which involves finding connected subgraphs
that optimize specific functions that model underlying processes about the data.
While there exist a number of heuristics, the work of \cite{cadena:sdm17} gives the
first rigorous methods for optimizing most scan statistics using the color coding technique.
However, all these methods are sequential.
